Bahasa pemrograman C/C++ merupakan bahasa yang popular didalam
pengajaran pada computer sains maupun pada kalangan programmer yang
mengembangkan system software maupun aplikasi.

Bahasa C/C++ sifatnya portable, karena compilernya tersedia hampir pada
semua arsitektur computer maupun system operasi, sehingga investasi
waktu dan tenaga yang anda lakukan dalam mempelajari bahasa pemrograman
ini memiliki nilai strategis yang sangat menjanjikan.

Bahasa C/C++ merupakan bahasa yang sangat ketat dalam pemakaian type
data maupun penulisannya yang case sensitif, hal ini berarti programmer
di tuntut kedisiplinannya dalam penulisan program.

Sesuatu fasilitas yang tersedia dalam C/C++ yang tidak ditemukan pada
bahasa pemrogaman lainnya adalah pointer, dengan pemanfaatan pointer
programmer dapat melakukan manipulasi memori secara langsung.

Dewasa ini beberapa bahasa yang memiliki syntax penulisan yang
menyerupai C/C++ adalah Java, Javascript dan PHP, yang artinya bahwa
kemampuan pemrograman dengan C/C++ akan mempermudah anda untuk
mempelajari bahasa modern seperti Java maupun C\# (dibaca C sharp).

Akhirnya penulis mengucapkan selamat belajar dan semoga buku ini dapat
memberi manfaat yang sebesarnya dalam pembelajaran mata kuliah C/C++
Programming.

\section*{Struktur buku ini}

\begin{description}
	\item[Bab 1. Mukadimah] Pada bab ini akan di perkenalkan mengenai
	 bahasa C++, mengenal Qt Creator, teknologi User Interface yang 
	 digunakan, Instalasi dan Struktur program C++ pada Qt Creator secara umum 
	\item[Bab 2. tipe data, identifier, Operator dan Control Flow] 
	pada bab ini akan di bahas tuntas mengenai apa itu tipe data C++, 
	Variabel dan konsanta, Statement, Operator dan Control Statement.
	\item[Bab 3. Array dan String] Pada bab ini akan di perkenalkan 
	macam-macam array dari array 1 dimensi sampai dengan array 
	multidimesi dan String berupa penggunaan string di dalam bahasa C++.
	\item[Bab 4. Fungsi] Pada bab ini akan mempelari tentang konsep dasar
	fungsi cara mendefinisikan fungsi dan medeklarasikanya, Hasil balik
	fungsi serta mempelajari ruang lingkup variabel dan pengiriman parameter.
	\item[Bab 5. Pointer dan Array] Pada bab ini akan di perkenalkan mengenai
	pointer di dalam komputer dan cara penggunaan dalam bahasa C++ kemudian
	tentang Array dan sedikit tentang reference.
	\item[Bab 6. Class dan Objek] Pada bab ini akan di paparkan secara gamblang
	mengenai pemrograman berorietasi objek dengan memperkenalkan struktur dasar
	dari OOP yakni Class dan objek.
	\item[Bab 7. Inheritance] Pada bab ini akan di perkenalakan tentang Inheritance
	yakni tentang pemakaian kode yang telah ada untuk di gunakan kembali.
	\item[Bab 8. Operator dan Operator Overloading] Pada bab awal kita sudah 
	mempelajari berbagai macam operator (+, -, /, >, <) yang dapat digunakan
	pada tipe data yang sudah ada di C++ seperti int, float, bool, dll. 
	Namun jika anda ingin menggunakan operator tersebut pada tipe data yang 
	Anda defnisikan sendiri seperti tipe data Class, maka anda dapat 
	menggunakan keyword operator.
	\item[Bab 9. Polymorphism] Pada bab ini akan mempelajari tentang Polymorphism
	yakni suatu konsep untuk merelasikan
diatara kelas-kelas C++ melalui overriding metode-metode virtual,
sehingga dengan demikian satu tipe kelas dapat konversikan menjadi kelas
lain. Aspek penting pertama dalam Pewarisan (\emph{Inheritance}) adalah
pengoganisasian kelas yang mengijinkan kelas lain berbagi program dan
data (\emph{code reuse}) sehingga progam tidak harus dibangun ulang dari
awal. Aspek penting kedua dari Pewarisan (\emph{Inheritance}) adalah
pengelompokan fitur-fitur kelas yang serupa kedalam sebuah kelas dasar
(\emph{base class}) dan kemudian membuat kelas lain dengan cara
menurunkan kelas tersebut sehingga bentuk utama/ pokok dari kelas-kelas
turunan menjadi serupa dengan kelas dasar sedemikian rupa sehingga
\texttt{pointer} atau referensi bertipe kelas dasar dapat menerima nilai
berbagai macam bentuk objek bertipe kelas turunannya (berubah tipe
kelas) dan dapat mengeksekusi fitur-fitur yang serupa tesebut.
\item[Bab 10. Casting dan Database] Casting merupakan mekanisme dimana programmer dapat secara permanen atau
temporary mengubah interpretasi compiler terhadap suatu obyek. Perubahan
ini tidak benar-benar terjadi, namun hanya cara pandang compiler saja
yang diubah. Casting diimplementasikan dalam bentuk ``casting
operator''. Mengapa butuh casting? Dalam dunia pemrograman yang semuanya
jelas (strong type language) dan jika kita hanya menggunakan satu bahasa
pemrograman saja, seperti C++, maka kita tidak membutuhkan operator
casting. Namun kenyataannya pada dunia nyata yang kita hadapi, banyak
bahasa pemrograman, banyak vendor-vendor berbeda-beda sehingga kode /
modul yg dihasilkan jg berbeda-beda. Hal ini menyebabkan
compiler-compiler bahasa pemrograman tertentu, termasuk C++ juga harus
diubah interpretasinya dengan cara lain sehingga mampu melakukan
kompilasi dan menghasilkan hasil yang kompatibel.
\item[Bab 11. GUI] Pada bab ini akan memamparkan bagaimana membuat sebuah aplikasi GUI 
sederhana dengan mudah menggunakan Qt Creator . Aplikasi sederhana dengan
fungsi dasar tombol dan label (untuk gambar). Jika tombol pertama di klik, maka
gamabr A muncul. Jika tombol kedua di klik, maka gambar B muncul. Dengan
contoh dasar tersebut di harapakan mamou memahamai dasar-dasar dari sebuah
pemrograman GUI di Qt Creator yakni signal dan slot.
\item[Bab 12. File, Stream dan XML] Pada bab ini kita akan membahas 
tentang beberapa class khusus pada Qt Framework yang digunakan untuk
bekerja dengan File dan dokumen XML.
\item[Bab 13.]
\item[Bab 14. Library] Pada bab ini akan mempelajari mengenai Qt SDK
menyediakan beberapa class library yang dapat anda gunakan untuk
mempercepat pembuatan program, misalnya library untuk membuat GUI
(Graphical User Interface), network programming, dan library untuk
bekerja dengan XML.

\end{description}

\section*{Apakah buku ini telah selesai?}

Belum, buku ini jauh dari sempurna, maka dari itu jika Anda tertarik 
untuk mengembangkan atau menambahkan konten maka anda bisa fork repository
saya di \url{https://github.com/turahe/cpp-qt}.

\section*{Kontribusi}

Jika Anda ingin berkontribusi dalam penyusunan buku ini Anda dapat
mengirimkan beberapa artikel pendukung ke email saya 

\section*{Dukungan}

