\textbf{📋 Apa yang akan dipelajari}

Pada bab ini kita akan mempelajari tentang Qt Library dan class-class penting yang disediakan oleh Qt:

\begin{itemize}
\item Qt Core Module dan class-class dasarnya
\item QObject dan automatic memory management
\item QString dan manipulasi string
\item Collection classes (QList, QMap, QStack, QQueue)
\item Iterator dan cara mengakses data
\end{itemize}

\minitoc

\section{📚 Pengenalan Qt Library}

\subsection{Apa itu Qt Library?}

Qt Library adalah kumpulan class-class C++ yang disediakan oleh Qt untuk mempercepat pengembangan aplikasi. Library ini menyediakan berbagai fitur siap pakai seperti:

\begin{itemize}
\item \textbf{GUI Components} - Widget untuk membuat antarmuka grafis
\item \textbf{Network Programming} - Class untuk komunikasi jaringan
\item \textbf{Database Access} - Class untuk mengakses database
\item \textbf{File I/O} - Class untuk membaca dan menulis file
\item \textbf{Data Structures} - Collection classes untuk menyimpan data
\end{itemize}

\subsection{Keunggulan Qt Library}

\begin{itemize}
\item \textbf{Cross-platform} - Sama di Windows, Mac, Linux
\item \textbf{Type-safe} - Mencegah kesalahan tipe data
\item \textbf{Memory Management} - Otomatis mengelola memory
\item \textbf{Rich Features} - Banyak fitur siap pakai
\item \textbf{Documentation} - Dokumentasi lengkap dan contoh
\end{itemize}

\subsection{Qt Core Module}

Qt Core Module adalah library dasar yang dibutuhkan oleh setiap aplikasi Qt. Module ini berisi:

\begin{itemize}
\item \textbf{Basic Data Types} - QString, QByteArray, QChar
\item \textbf{Data Structures} - QList, QVector, QHash, QMap
\item \textbf{I/O Classes} - QIODevice, QTextStream, QFile
\item \textbf{Object System} - QObject, QCoreApplication
\item \textbf{Threading} - QThread untuk pemrograman multi-thread
\end{itemize}

\section{🏗️ QObject - Base Class Penting}

\subsection{Apa itu QObject?}

QObject adalah base class yang sangat penting dalam Qt. Hampir semua class Qt diturunkan dari QObject. Class ini menyediakan:

\begin{itemize}
\item \textbf{Automatic Memory Management} - Otomatis menghapus objek
\item \textbf{Signal-Slot Mechanism} - Komunikasi antar objek
\item \textbf{Object Tree} - Hierarki parent-child
\item \textbf{Event System} - Sistem event handling
\item \textbf{Property System} - Sistem properti dinamis
\end{itemize}

\subsection{Class Hierarchy Qt}

\begin{center}
\begin{tabular}{|l|l|l|}
\hline
\textbf{Class} & \textbf{Inherits from} & \textbf{Purpose} \\
\hline
QObject & - & Base class untuk semua objek Qt \\
\hline
QWidget & QObject & Base class untuk GUI widgets \\
\hline
QString & - & Class untuk string handling \\
\hline
QList & - & Template class untuk list \\
\hline
QMap & - & Template class untuk map \\
\hline
\end{tabular}
\end{center}

\subsection{Keuntungan Menggunakan QObject}

\begin{enumerate}
\item \textbf{Memory Safety} - Tidak perlu manual delete
\item \textbf{Object Tree} - Parent-child relationship
\item \textbf{Signal-Slot} - Event-driven programming
\item \textbf{Properties} - Dynamic property system
\item \textbf{Events} - Event handling system
\end{enumerate}

\section{🧠 Automatic Memory Management}

\subsection{Masalah Memory Management}

Dalam C++ tradisional, programmer harus manual mengelola memory:
\begin{itemize}
\item \textbf{Memory Leak} - Lupa delete objek
\item \textbf{Dangling Pointer} - Pointer ke objek yang sudah dihapus
\item \textbf{Double Delete} - Menghapus objek yang sudah dihapus
\item \textbf{Complex Management} - Sulit dalam aplikasi besar
\end{itemize}

\subsection{QObject Memory Management}

QObject menyediakan automatic memory management:

\begin{itemize}
\item \textbf{Parent-Child Relationship} - Objek child dihapus otomatis
\item \textbf{Object Tree} - Hierarki objek yang terorganisir
\item \textbf{Automatic Cleanup} - Tidak perlu manual delete
\item \textbf{Stack-based Parent} - Parent disimpan di stack
\end{itemize}

\subsection{Contoh Memory Management}

\subsubsection{❌ Tanpa QObject (Manual Management)}

\begin{enumerate}
\item Buka Qt Creator dan buat project Qt Console Application baru
\item Tulis kode berikut:

\lstinputlisting[language=c++, caption=Alokasi memory dinamis tanpa QObject, label=contoh11-1]{../code/11-library-contoh11-1.c++}

\item Jalankan program dan perhatikan output:

\begin{lcverbatim}
22002321 : 8 kar
22002322 : 8 kar
22002323 : 8 kar
22002323 : 8 kar
destroy object
destroy object
destroy object
\end{lcverbatim}
\end{enumerate}

\textbf{Analisis Program:}
\begin{itemize}
\item Class Mahasiswa tidak diturunkan dari QObject
\item Harus manual delete setiap objek
\item Risk memory leak jika lupa delete
\item Sulit dalam aplikasi kompleks
\end{itemize}

\subsubsection{✅ Dengan QObject (Automatic Management)}

\begin{enumerate}
\item Buka Qt Creator dan buat project Qt Console Application baru
\item Tulis kode berikut:

\lstinputlisting[language=c++, caption=Alokasi memory dinamis dengan QObject, label=contoh11-2]{../code/11-library-contoh11-2.c++}

\item Jalankan program dan perhatikan output:

\begin{lcverbatim}
"22002321"  :  8  kar
"22002322"  :  8  kar
"22002323"  :  8  kar
"22002323"  :  8  kar
\end{lcverbatim}
\end{enumerate}

\textbf{Analisis Program:}
\begin{itemize}
\item Class Mahasiswa diturunkan dari QObject
\item Memory dihapus otomatis saat program selesai
\item Tidak perlu manual delete
\item Lebih aman dan mudah
\end{itemize}

\subsection{Parent-Child Relationship}

QObject menggunakan sistem parent-child untuk memory management:

\begin{itemize}
\item \textbf{Parent} - Objek yang memiliki child
\item \textbf{Child} - Objek yang dimiliki parent
\item \textbf{Automatic Deletion} - Child dihapus saat parent dihapus
\item \textbf{Object Tree} - Hierarki objek yang terorganisir
\end{itemize}

\section{📝 QString - String Handling}

\subsection{Mengapa Menggunakan QString?}

QString adalah class untuk menangani string dalam Qt. Keunggulannya:

\begin{itemize}
\item \textbf{Unicode Support} - Mendukung semua bahasa
\item \textbf{Rich Features} - Banyak method untuk manipulasi
\item \textbf{Cross-platform} - Sama di semua platform
\item \textbf{Type-safe} - Mencegah kesalahan tipe data
\item \textbf{Performance} - Optimized untuk Qt
\end{itemize}

\subsection{Perbandingan QString vs std::string}

\begin{center}
\begin{tabular}{|l|l|l|}
\hline
\textbf{Aspek} & \textbf{QString} & \textbf{std::string} \\
\hline
Unicode & Full support & Limited \\
\hline
Qt Integration & Native & Requires conversion \\
\hline
Platform & Cross-platform & Platform dependent \\
\hline
Features & Rich & Basic \\
\hline
Performance & Optimized & Standard \\
\hline
\end{tabular}
\end{center}

\subsection{Basic String Operations}

\subsubsection{Checking String Properties}

\begin{enumerate}
\item Buka Qt Creator dan buat project Qt Console Application baru
\item Tulis kode berikut:

\lstinputlisting[language=c++, caption=Cek apakah nilai QString Null atau Empty, label=contoh11-3]{../code/11-library-contoh11-3.c++}

\item Jalankan program dan perhatikan output:

\begin{lcverbatim}
"Erick Kurniawan"
Ukuran string  15
test not null
test empty
testing null
\end{lcverbatim}
\end{enumerate}

\textbf{Method yang Digunakan:}
\begin{itemize}
\item \textbf{size()} - Mendapatkan panjang string
\item \textbf{isNull()} - Cek apakah string belum diinisialisasi
\item \textbf{isEmpty()} - Cek apakah string kosong
\item \textbf{length()} - Mendapatkan jumlah karakter
\end{itemize}

\subsubsection{String Extraction}

\begin{enumerate}
\item Buka Qt Creator dan buat project Qt Console Application baru
\item Tulis kode berikut:

\lstinputlisting[language=c++, caption=Menggunakan Fungsi Left Mid Right, label=contoh11-4]{../code/11-library-contoh11-4.c++}

\item Jalankan program dan perhatikan output:

\begin{lcverbatim}
firstName :  "Erick"
lastName :  "Kurniawan"
midName :  "Kurni"
\end{lcverbatim}
\end{enumerate}

\textbf{Method yang Digunakan:}
\begin{itemize}
\item \textbf{left(n)} - Ambil n karakter dari kiri
\item \textbf{right(n)} - Ambil n karakter dari kanan
\item \textbf{mid(pos, n)} - Ambil n karakter dari posisi pos
\item \textbf{mid(pos)} - Ambil semua karakter dari posisi pos
\end{itemize}

\subsubsection{String Concatenation}

\begin{enumerate}
\item Buka Qt Creator dan buat project Qt Console Application baru
\item Tulis kode berikut:

\lstinputlisting[language=c++, caption=Menggabungkan String, label=contoh11-5]{../code/11-library-contoh11-5.c++}

\item Jalankan program dan perhatikan output:

\begin{lcverbatim}
Nama :  "Pak. Slamet ,BA"
"Pak. Slamet ,BA ,SE"
\end{lcverbatim}
\end{enumerate}

\textbf{Method yang Digunakan:}
\begin{itemize}
\item \textbf{append(str)} - Tambah string di akhir
\item \textbf{prepend(str)} - Tambah string di awal
\item \textbf{insert(pos, str)} - Sisip string di posisi tertentu
\item \textbf{operator+()} - Gabung string dengan operator +
\end{itemize}

\subsubsection{String Reversal}

\begin{enumerate}
\item Buka Qt Creator dan buat project Qt Console Application baru
\item Tulis kode berikut:

\lstinputlisting[language=c++, caption=Membalik String, label=contoh11-6]{../code/11-library-contoh11-6.c++}

\item Jalankan program dan perhatikan output:

\begin{lcverbatim}
Balik :  "dihcaW ruN"
\end{lcverbatim}
\end{enumerate}

\textbf{Method yang Digunakan:}
\begin{itemize}
\item \textbf{operator[]()} - Akses karakter dengan index
\item \textbf{size()} - Mendapatkan panjang string
\item \textbf{Manual Loop} - Loop untuk membalik string
\end{itemize}

\subsection{Advanced String Operations}

\subsubsection{String Conversion}

\begin{itemize}
\item \textbf{toStdString()} - Konversi ke std::string
\item \textbf{fromStdString()} - Konversi dari std::string
\item \textbf{toInt()} - Konversi ke integer
\item \textbf{toDouble()} - Konversi ke double
\item \textbf{toLower()} - Konversi ke huruf kecil
\item \textbf{toUpper()} - Konversi ke huruf besar
\end{itemize}

\subsubsection{String Searching}

\begin{itemize}
\item \textbf{contains()} - Cek apakah string mengandung substring
\item \textbf{indexOf()} - Cari posisi substring
\item \textbf{lastIndexOf()} - Cari posisi terakhir substring
\item \textbf{startsWith()} - Cek apakah string dimulai dengan substring
\item \textbf{endsWith()} - Cek apakah string diakhiri dengan substring
\end{itemize}

\section{📦 Collection Classes}

\subsection{Apa itu Collection Classes?}

Collection classes adalah class-class untuk menyimpan dan mengelola kumpulan data. Qt menyediakan berbagai collection yang type-safe dan optimized.

\subsection{Jenis Collection Classes}

\begin{itemize}
\item \textbf{QList} - Dynamic array, paling umum digunakan
\item \textbf{QVector} - Array dengan akses cepat
\item \textbf{QLinkedList} - Linked list untuk operasi insert/delete
\item \textbf{QMap} - Key-value pair, terurut
\item \textbf{QHash} - Key-value pair, tidak terurut, lebih cepat
\item \textbf{QSet} - Set of unique values
\item \textbf{QStack} - LIFO (Last In First Out)
\item \textbf{QQueue} - FIFO (First In First Out)
\end{itemize}

\subsection{QList - Dynamic Array}

QList adalah collection yang paling sering digunakan. Keunggulannya:

\begin{itemize}
\item \textbf{Dynamic Size} - Ukuran berubah otomatis
\item \textbf{Fast Access} - Akses elemen dengan index
\item \textbf{Type-safe} - Mencegah kesalahan tipe data
\item \textbf{Rich API} - Banyak method untuk manipulasi
\end{itemize}

\subsubsection{Basic QList Operations}

\begin{enumerate}
\item Buka Qt Creator dan buat project Qt Console Application baru
\item Tulis kode berikut:

\lstinputlisting[language=c++, caption=Menggunakan QList, label=contoh11-7]{../code/11-library-contoh11-7.c++}

\item Jalankan program dan perhatikan output:

\begin{lcverbatim}
"Erick"
"Erick"
"Anton"
"Katon"
"Budi"
\end{lcverbatim}
\end{enumerate}

\textbf{Method yang Digunakan:}
\begin{itemize}
\item \textbf{append()} - Tambah elemen di akhir
\item \textbf{prepend()} - Tambah elemen di awal
\item \textbf{insert()} - Sisip elemen di posisi tertentu
\item \textbf{foreach} - Loop untuk mengakses semua elemen
\end{itemize}

\subsection{Iterator - Mengakses Data}

Iterator adalah cara untuk mengakses data dalam collection secara berurutan.

\subsubsection{QListIterator - Read-only Iterator}

\begin{enumerate}
\item Buka Qt Creator dan buat project Qt Console Application baru
\item Tulis kode berikut:

\lstinputlisting[language=c++, caption=Menggunakan object Iterator, label=contoh11-8]{../code/11-library-contoh11-8.c++}

\item Jalankan program dan perhatikan output:

\begin{lcverbatim}
12
24
36
48
60
12
24
36
48
60
\end{lcverbatim}
\end{enumerate}

\textbf{Method Iterator:}
\begin{itemize}
\item \textbf{hasNext()} - Cek apakah masih ada elemen berikutnya
\item \textbf{next()} - Pindah ke elemen berikutnya
\item \textbf{hasPrevious()} - Cek apakah masih ada elemen sebelumnya
\item \textbf{previous()} - Pindah ke elemen sebelumnya
\end{itemize}

\subsubsection{QMutableListIterator - Modifiable Iterator}

\begin{enumerate}
\item Buka Qt Creator dan buat project Qt Console Application baru
\item Tulis kode berikut:

\lstinputlisting[language=c++, caption=Menggunakan Iterator untuk memodifikasi data di list, label=contoh11-9]{../code/11-library-contoh11-9.c++}

\item Jalankan program dan perhatikan output:

\begin{lcverbatim}
"update data.."
"Anton"
"Katon"
"update data.."
\end{lcverbatim}
\end{enumerate}

\textbf{Method Mutable Iterator:}
\begin{itemize}
\item \textbf{setValue()} - Ubah nilai elemen saat ini
\item \textbf{insert()} - Sisip elemen baru
\item \textbf{remove()} - Hapus elemen saat ini
\end{itemize}

\subsection{Adding Data to List}

Ada berbagai cara untuk menambahkan data ke QList:

\begin{enumerate}
\item Buka Qt Creator dan buat project Qt Console Application baru
\item Tulis kode berikut:

\lstinputlisting[language=c++, caption=Beberapa cara menambahkan data ke list, label=contoh11-10]{../code/11-library-contoh11-10.c++}

\item Jalankan program dan perhatikan output:

\begin{lcverbatim}
"anton"
"erick"
"budi"
"katon"
"naren"
\end{lcverbatim}
\end{enumerate}

\textbf{Cara Menambah Data:}
\begin{itemize}
\item \textbf{append()} - Tambah di akhir list
\item \textbf{prepend()} - Tambah di awal list
\item \textbf{insert()} - Sisip di posisi tertentu
\item \textbf{operator<<()} - Operator untuk append
\item \textbf{push_back()} - Tambah di akhir (STL style)
\end{itemize}

\subsection{QStringList - Special String List}

QStringList adalah class khusus untuk list of strings. Class ini diturunkan dari QList dan memiliki method tambahan untuk manipulasi string.

\begin{enumerate}
\item Buka Qt Creator dan buat project Qt Console Application baru
\item Tulis kode berikut:

\lstinputlisting[language=c++, caption=Menggunakan QStringList, label=contoh11-11]{../code/11-library-contoh11-11.c++}

\item Jalankan program dan perhatikan output:

\begin{lcverbatim}
"Jogjakarta,Jakarta,Bandung,Semarang"
"Jogjakarta"
"Jakarta"
"Bandung"
"Semarang"
"Jogjaaakaaartaaa"
"Jaaakaaartaaa"
"Baaandung"
"Semaaaraaang"
\end{lcverbatim}
\end{enumerate}

\textbf{Method QStringList:}
\begin{itemize}
\item \textbf{split()} - Pecah string berdasarkan delimiter
\item \textbf{join()} - Gabung list menjadi string
\item \textbf{replaceInStrings()} - Ganti substring dalam semua elemen
\item \textbf{filter()} - Filter elemen berdasarkan kondisi
\end{itemize}

\section{🗂️ Stack dan Queue}

\subsection{Stack - LIFO (Last In First Out)}

QStack adalah collection yang menggunakan prinsip LIFO. Elemen terakhir yang masuk akan keluar pertama.

\subsection{Queue - FIFO (First In First Out)}

QQueue adalah collection yang menggunakan prinsip FIFO. Elemen pertama yang masuk akan keluar pertama.

\begin{enumerate}
\item Buka Qt Creator dan buat project Qt Console Application baru
\item Tulis kode berikut:

\lstinputlisting[language=c++, caption=Menggunakan Stack dan Queue, label=contoh11-12]{../code/11-library-contoh11-12.c++}

\item Jalankan program dan perhatikan output:

\begin{lcverbatim}
Stack LIFO : 
"budi"
"katon"
"anton"
"17rick"
Queue FIFO : 
"17rick"
"anton"
"katon"
"budi"
\end{lcverbatim}
\end{enumerate}

\textbf{Perbedaan Stack dan Queue:}

\begin{center}
\begin{tabular}{|l|l|l|}
\hline
\textbf{Aspek} & \textbf{QStack (LIFO)} & \textbf{QQueue (FIFO)} \\
\hline
Prinsip & Last In First Out & First In First Out \\
\hline
Add & push() & enqueue() \\
\hline
Remove & pop() & dequeue() \\
\hline
Peek & top() & head() \\
\hline
Use Case & Undo/Redo & Task Queue \\
\hline
\end{tabular}
\end{center}

\section{🗺️ QMap - Key-Value Pair}

\subsection{Apa itu QMap?}

QMap adalah collection yang menyimpan data dalam bentuk key-value pair. Setiap elemen memiliki key unik dan value yang terkait.

\subsection{Keunggulan QMap}

\begin{itemize}
\item \textbf{Key-Value Pair} - Data terorganisir dengan key
\item \textbf{Ordered} - Data terurut berdasarkan key
\item \textbf{Fast Lookup} - Akses cepat dengan key
\item \textbf{Type-safe} - Mencegah kesalahan tipe data
\end{itemize}

\begin{enumerate}
\item Buka Qt Creator dan buat project Qt Console Application baru
\item Tulis kode berikut:

\lstinputlisting[language=c++, caption=Menggunakan QMap, label=contoh11-13]{../code/11-library-contoh11-13.c++}

\item Jalankan program dan perhatikan output:

\begin{lcverbatim}
erick age :  29
menampilkan semua data yg ada di map :
"anton"  :  29
"erick"  :  29
"katon"  :  42
Mengakses data menggunakan iterator
"anton"  :  29
"erick"  :  29
"katon"  :  42
\end{lcverbatim}
\end{enumerate}

\textbf{Method QMap:}
\begin{itemize}
\item \textbf{insert(key, value)} - Tambah key-value pair
\item \textbf{operator[]()} - Akses value dengan key
\item \textbf{contains()} - Cek apakah key ada
\item \textbf{keys()} - Dapatkan semua key
\item \textbf{values()} - Dapatkan semua value
\item \textbf{remove()} - Hapus key-value pair
\end{itemize}

\section{🔧 Best Practices Qt Library}

\subsection{Tips Menggunakan Qt Library}

\begin{enumerate}
\item \textbf{Gunakan QString} - Jangan campur dengan std::string
\item \textbf{Pilih Collection yang Tepat} - QList untuk umum, QMap untuk key-value
\item \textbf{Gunakan Iterator} - Untuk akses berurutan dan modifikasi
\item \textbf{Memory Management} - Manfaatkan QObject automatic memory
\item \textbf{Type Safety} - Gunakan template dengan tipe yang tepat
\end{enumerate}

\subsection{Kesalahan Umum}

\begin{itemize}
\item \textbf{Mixing STL and Qt} - Campur std::string dengan QString
\item \textbf{Wrong Collection} - Gunakan QList untuk key-value
\item \textbf{Manual Memory} - Lupa manfaatkan QObject
\item \textbf{Type Mismatch} - Masukkan tipe data salah ke collection
\item \textbf{Inefficient Access} - Tidak gunakan iterator untuk loop
\end{itemize}

\section{📚 Referensi dan Bacaan Lanjutan}

Qt Library adalah topik yang luas dan terus berkembang. Untuk pemahaman yang lebih mendalam, pembaca dapat merujuk pada:

\begin{itemize}
\item \textbf{Qt Documentation} - Dokumentasi resmi Qt\footnote{Qt Company. (2023). "Qt Core Module". https://doc.qt.io/qt-6/qtcore-index.html}
\item \textbf{C++ GUI Programming with Qt 4} oleh Jasmin Blanchette\footnote{Blanchette, J., \& Summerfield, M. (2006). "C++ GUI Programming with Qt 4". Prentice Hall.}
\item \textbf{Qt 5 C++ GUI Programming Cookbook} oleh Lee Zhi Eng\footnote{Eng, L. Z. (2016). "Qt 5 C++ GUI Programming Cookbook". Packt Publishing.}
\item \textbf{Qt Core Module} - Module dasar Qt\footnote{Qt Company. (2023). "Qt Core Module". https://doc.qt.io/qt-6/qtcore-index.html}
\end{itemize}

\section{🎉 Kesimpulan}

Qt Library menyediakan class-class yang powerful dan type-safe untuk pengembangan aplikasi. Dengan memahami QObject, QString, dan collection classes, Anda dapat membuat aplikasi yang robust dan maintainable.

\begin{center}
\textbf{Selamat! Anda telah menguasai dasar-dasar Qt Library} 📚
\end{center}

\vspace{1cm}

\begin{center}
\textit{--- Bab selanjutnya: File, Stream, dan XML ---}
\end{center}
