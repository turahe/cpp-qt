\section*{Dasar Pemrograman C++}

Bagian ini akan membahas fondasi pemrograman C++ yang merupakan langkah awal yang penting sebelum mempelajari konsep-konsep lanjutan. Dalam bagian ini, Anda akan mempelajari:

\begin{itemize}
\item \textbf{Pengenalan C++ dan Qt Creator} - Sejarah, keunggulan, dan cara menggunakan IDE
\item \textbf{Tipe Data dan Operator} - Jenis data, variabel, dan operasi dasar
\item \textbf{Control Statement} - Struktur kontrol program (if, switch, loop)
\item \textbf{Array dan String} - Koleksi data dan manipulasi teks
\item \textbf{Fungsi} - Modularisasi program dan reusability
\item \textbf{Pointer dan References} - Manajemen memory dan referensi
\item \textbf{Debugging} - Teknik menemukan dan memperbaiki bug
\end{itemize}

\begin{quote}
\textit{"Pemrograman yang baik dimulai dari pemahaman yang kuat terhadap dasar-dasar bahasa pemrograman."}
\end{quote}

\vspace{1cm}

\begin{center}
\textbf{--- Mari kita mulai perjalanan belajar C++ dan Qt ---}
\end{center}
