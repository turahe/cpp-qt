\textbf{Agenda}

Pada chapter ini kita akan membahas tentang beberapa class khusus pada
Qt Framework yang digunakan untuk bekerja dengan File dan dokumen XML.
Adapun materi yang akan dibahas pada HOL ini adalah:

\minitoc

\section{Bekerja dengan Paths}\label{bekerja-dengan-paths}

\index{Qdir}QDir digunakan untuk bekerja dengan paths dan drives pada aplikasi Qt.
QDir memiliki beberapa static method yang memudahkan anda bekerja dengan
file sistem. Misal \texttt{QDir::current()} dapat digunakan untuk
mengembalikan QDir dari direktori kerja anda, \texttt{QDir::home()} akan
mengembalikan QDir dari home direktori pengguna, \texttt{QDir::root()}
akan mengembalikan root direktori, dan \texttt{QDir::drives()} akan
mengembalikan objek \texttt{QList\textless{}QFileInfo\textgreater{}}
yang mewakili root dari semua drive yang ada. Objek QFileInfo menyimpan
informasi tentang file dan direktori, ada beberapa method penting yang
sering digunakan yaitu:

\begin{itemize}

\item
  \texttt{isDir()}, \texttt{isFile()}, dan \texttt{isSymLink()} akan
  mengembalikan nilai true jika objek yang dicek berupa direktori, file,
  atau symbolic link (shortcut pada window).
\item
  \texttt{dir()} dan \texttt{absoluteDir()} akan mengembalikan
  \texttt{QDir} yang mengandung informasi dari objek file. Method
  \texttt{dir()} akan mengembalikan direktori relatif dari direktori
  aktif, dan method \texttt{absoluteDir()} mengembalikan path direktori
  yang dimulai dari root.
\item
  \texttt{exists()} akan menghasilkan nilai true jika objek tersebut
  ada.
\item
  \texttt{isHidden()}, \texttt{isReadable()}, \texttt{isWritable()}, dan
  \texttt{isExecutable()} mengembalikan status hak akses dari file.
\item
  \texttt{fileName()} akan mengembalikan \index{QString}QString berupa nama file tanpa
  path.
\item
  filePath() akan mengembalikan QString berupa nama file beserta path,
  path dapat bersifat relatif terhadap direktori aktif.
\item
  \texttt{absoluteFilePath()} akan mengembalikan QString berupa nama
  file beserta path, path diawali dari drive root.
\item
  \texttt{completeBaseName()} and \texttt{completeSuffix()} akan
  mengembalikan QString berupa nama file dan ekstensi (suffix).
\end{itemize}

Program dibawah ini akan menunjukan cara penggunaan QDir untuk mengambil
informasi direktori yang ada di komputer anda.

\subsubsection*{Menampilkan daftar drives dari root directories.}

\begin{enumerate}

\item
  Buka Qt Creator dan buat project Qt Console Application baru dengan
  nama Contoh \ref{contoh12-1}, kemudian tulis kode berikut.

\lstinputlisting[language=c++, caption=Menampilkan daftar drives dari root directories., label=contoh12-1]{../code/contoh12-1.cpp}

\item
  Kemudian jalankan kode diatas dengan menekan tombol Ctrl+R, maka akan
  ditampilkan output sebagai berikut.
  \begin{lcverbatim}
Drive :  "C:/"
  "MATLAB"
  "PerfLogs"
  "Program Files"
  "Program Files (x86)"
  "Qt"
  "system.sav"
  "Users"
  "Windows"
Drive :  "D:/"
  "Music"
  "My Web Sites"
  "Photo"
  "Portable"
  "Referensi"
  "Software"
  "video"
  "webs"

\end{lcverbatim}
\end{enumerate}

\textbf{Keterangan:}:

\begin{itemize}

\item
  Static method \texttt{QDir::drives()} akan mengembalikan collection
  berupa \texttt{QList\textless{}FileInfo\textgreater{}} yang berisi
  semua drive yang ada pada komputer.
\item
  Untuk membaca semua drive beserta semua folder /direktori didalamnya
  anda dapat menggunakan foreach.
\item
  Method \texttt{setFilter(QDir::Dirs)} artinya yang akan ditampilkan
  hanya direktori saja, tidak file atau symbolic link, untuk menampilkan
  semua (drive, direktori, file) anda dapat menggunakan
  \texttt{QDir::AllEntries}.
\item
  Output dari program tersebut adalah daftar drive pada komputer anda
  beserta dengan folder yang ada didalamnya.
\end{itemize}

Dibawah ini adalah daftar kriteria yang dapat digunakan untuk melakukan
filter.

\begin{itemize}

\item
  QDir::Dirs: Lists directories.
\item
  QDir::AllDirs: Lists all directories.
\item
  QDir::Files: Lists files.
\item
  QDir::Drives: Lists drives.
\item
  QDir::NoSymLinks: tidak menampilkan symbolic links.
\item
  QDir::NoDotAndDotDot: tidak menampilkan special entries .
\item
  QDir::AllEntries: Lists directories, files, drives, dan symbolic
  links.
\item
  QDir::Readable: Lists readable files, harus dikombinasikan dengan
  Files atau Dirs.
\item
  QDir::Writeable: Lists writable files. harus dikombinasikan dengan
  Files atau Dirs.
\item
  QDir::Executable: Lists executable files. harus dikombinasikan dengan
  Files atau Dirs.
\item
  QDir::Modified: Lists files yang sudah dimodifikasi.
\item
  QDir::Hidden: Lists files yang sifatnya hidden.
\item
  QDir::System: Lists system files.
\item
  QDir::CaseSensitive: filter name harus case sensitive jika file sistem
  case sensitive.
\end{itemize}

\section{Bekerja dengan Files}\label{bekerja-dengan-files}

Anda dapat menggunakan QDir untuk mengambil informasi file dan \index{QFileInfo}QFileInfo
untuk mengambil informasi file yang lebih lengkap, untuk proses
manipulasi yang lebih jauh lagi seperti membuka, membaca, dan
memodifikasi file anda harus menggunakan class QFile.

Agar lebih jelas bagaimana cara menggunakan \index{QFile}QFile untuk membuka file,
anda dapat mengerjakan Contoh \ref{contoh12-2} dibawah ini.

\subsubsection*{Contoh  Memeriksa apakah file ada dan bisa diakses.}

\begin{enumerate}

\item
  Buka Qt Creator dan buat project Qt Console Application baru dengan
  nama contoh \ref{contoh12-2}, kemudian tulis kode berikut.

\lstinputlisting[language=c++, caption=Memeriksa apakah file ada dan bisa diakses, label=contoh12-2]{../code/contoh12-2.cpp}
\item
  Kemudian jalankan kode diatas dengan menekan tombol Ctrl+R, maka akan
  ditampilkan output sebagai berikut.
  
  \begin{lcverbatim}
File :  "testfile.txt"  tidak ditemukan
  \end{lcverbatim}
\item
  Output diatas mempunyai arti bahwa anda belum membuat file dengan nama
  ``testfile.txt'', agar program diatas berhasil dijalankan buat file
  ``testfile.txt'' didalam folder Contoh build-\ref{contoh12-2}-Desktop\_Qt\_5\_5\_0\_MinGW\_32bit-Debug.
   Kemudian  jalankan kembali programnya, maka akan tampil output berikut.
  
  \begin{lcverbatim}
File berhasil dibuka !
  \end{lcverbatim}
  
\end{enumerate}

\textbf{Keterangan:}

\begin{itemize}

\item
  \texttt{Method\ exists()} digunakan untuk mengecek apakah file yang
  ada atau tidak. Jika file tidak ditemukan maka program akan keluar dan
  menampilkan output program tidak ditemukan.
\item
  Method \texttt{open()} digunakan untuk membuka file, permision yang
  digunakan pada program diatas adalah WriteOnly. Jika file tidak dapat
  dibuka maka akan keluar pesan tidak dapat membuka file.
\end{itemize}

\section{Bekerja dengan Stream}\label{bekerja-dengan-stream}

Setelah anda membuka file akan lebih mudah untuk mengakses file tersebut
menggunakan stream class. Qt hadir dengan 2 macam stream class, satu
untuk teks file dan satu lagi untuk binary file. Dengan menggunakan
stream untuk mengakses file anda dapat menggunakan operator
\texttt{\textless{}\textless{}} dan
\texttt{\textgreater{}\textgreater{}} untuk menulis dan membaca data
dari file.

\subsection{Text Stream}\label{text-stream}

Untuk membuat text stream pada file, buat objek QFile dan buka file
seperti biasa, disarankan jika anda menggunakan parameter
\texttt{QIODevice::Text} dan hak akses \texttt{QIODevice::ReadOnly}.
Agar lebih jelas bagaimana penggunaan stream buatlah contoh program pada
contoh \ref{contoh12-3} dibawah ini.

\subsubsection*{Menggunakan Stream untuk membaca file.}

\begin{enumerate}

\item
  Buka Qt Creator dan buat project Qt Console Application baru dengan
  nama contoh \ref{contoh12-3}, kemudian tulis kode berikut.

\lstinputlisting[language=c++, caption=Menggunakan Stream untuk membaca file, label=contoh12-3]{../code/contoh12-3.cpp}
\item
  Buat file teks pada alamat tertentu (pada contoh \ref{contoh12-3} diatas di drive
  \texttt{D:\textbackslash{}sample.txt}), masukan sembarang teks kedalam
  file tersebut.
\item
  Kemudian jalankan kode diatas dengan menekan tombol Ctrl+R, maka akan
  ditampilkan output sebagai berikut.
  
  \begin{lcverbatim}
Buku Pemrograman C++
  \end{lcverbatim}
\end{enumerate}

\textbf{Keterangan:}

\begin{itemize}

\item
  Objek \index{QtextStream}\texttt{QTextStream} digunakan jika anda ingin menggunakan
  stream untuk mengakses file text.
\item
  Untuk membaca semua data yang ada pada file text gunakan method
  readAll().
\item
  Untuk membaca file baris demi baris dapat digunakan method readLine()
  yang dijalankan didalam loop, method atEnd() digunakan untuk memeriksa
  apakah sudah sampai akhir file.
\end{itemize}

\subsection{Data Stream}\label{data-stream}

Pada beberapa kasus tertentu mungkin anda tidak dapat menggunakan file
text untuk menyimpan data. Misalnya anda harus bekerja dengan file yang
mempunyai format bukan teks, atau anda membutuhkan ukuran penyimpanan
yang lebih kecil daripada menggunakan file teks. Dengan menyimpan data
kedalam machine-readable seperti file biner ukuran file bisa menjadi
lebih kecil dibandingkan dengan menyimpan data kedalam human-readable
format seperti file teks.

Untuk membaca file biner anda dapat menggunakan objek \index{QDataStream}QDataStream.
Program dibawah ini menunjukan penggunaan objek QDataStream untuk
mengakses file biner.

\subsubsection*{Contoh  Menggunakan Data Stream.}

\begin{enumerate}

\item
  Buka Qt Creator dan buat project Qt Console Application baru dengan
  nama Contoh \ref{contoh12-4}
\item
  Buka file Contoh \ref{contoh12-4}.pro untuk menambahkan library \index{GUI}GUI karena pada
  controh program ini digunakan class \index{Qcolor}QColor.

\begin{lstlisting}[language=c++, caption=file pro untuk membuka library Qt GUI]
#-------------------------------------------------
#
# Project created by QtCreator 2016-01-03T19:58:33
#
#-------------------------------------------------
QT += core
QT += gui
TARGET = Contoh 4
CONFIG += console
CONFIG -= app_bundle
TEMPLATE = app
SOURCES += main.cpp
\end{lstlisting}
\item
  Kemudian tambahkan kode berikut pada file main.cpp.

\begin{lstlisting}[language=c++, caption=Menggunakan Data Stream, label=contoh12-4]
#include <QtCore/QCoreApplication>
#include <QDebug>
#include <QDataStream>
#include <QList>
#include <QColor>
#include <QFile>
struct Warna
{
QString text;
QColor color;
};
QDataStream &operator << (QDataStream &stream, const Warna &data)
{
stream << data.text << data.color;
return stream;
}
QDataStream &operator >>(QDataStream &stream, Warna &data)
{
stream >> data.text;
stream >> data.color;
return stream;
}
void saveList()
{
QList<Warn list;
Warna data;
data.text = "Merah";
data.color = Qt::red;
list << data;
data.text = "Biru";
data.color = Qt::blue;
list << data;
data.text = "Kuning";
data.color = Qt::yellow;
list << data;
data.text = "Hijau";
data.color = Qt::green;
list << data;
QFile file( "datastream.dat" );
if( !file.open( QIODevice::WriteOnly ) )
return;
QDataStream stream( &file );
stream.setVersion( QDataStream::Qt_4_7);
stream << list;
file.close();
}
void loadList()
{
QList<Warn list;
QFile file( "datastream.dat" );
if( !file.open( QIODevice::ReadOnly ) )
return;
QDataStream stream(&file);
stream.setVersion(QDataStream::Qt_4_7);
stream >> list;
file.close();
foreach( Warna data, list )
qDebug() << data.text << "("
<< data.color.red() << ","
<< data.color.green() << ","
<< data.color.blue() << ")";
}
int main(int argc, char *argv[])
{
QCoreApplication a(argc, argv);
saveList();
loadList();
return a.exec();
}
\end{lstlisting}
\item
  Kemudian jalankan kode diatas dengan menekan tombol Ctrl+R, maka akan
  ditampilkan output sebagai berikut.
\end{enumerate}
\begin{lcverbatim}
	content
\end{lcverbatim}
\textbf{Keterangan:}

\begin{itemize}

\item
  Pada program diatas struct Warna adalah user define type yang dibuat
  sendiri, struct Warna memiliki dua member variabel yaitu text yang
  bertipe QString dan color yang bertipe QColor.
\item
  Untuk memasukan data biner (dengan tipe data Warna) kedalam stream
  buat operator \textless{}\textless{}
\item
  Untuk mengambil data biner dari stream buat operator
  \textgreater{}\textgreater{}
\item
  Method saveList() digunakan untuk membuat objek warna , memasukan
  objek tersebut kedalam list dan menuliskannya kedalam file
  datastream.dat.
\item
  Method loadList() digunakan untuk mengambil data stream dari file
  datastream.dat dan menampilkan data tersebut dengan cara membaca dari
  list.
\end{itemize}

\section{XML}\label{xml}

XML adalah meta-language yang dapat digunakan untuk menyimpan data
terstruktur berupa string atau teks file. Komponen dasar penyusun
dokumen XML adalah tag, attribute, dan teks. Contoh dokumen XML yang
sederhana ditunjukan pada listing dibawah ini.

\begin{lstlisting}[language=xml]
<document name="DocName">
<author name="AuthorName" />
Some text
</document>
\end{lstlisting}

Pada dokumen XML diatas document tag mengandung author tag dan teks . Tag document diawali
dengan \textless{}document\textgreater{} dan diakhiri dengan closing tag 
\textless{}\textfractionsolidus{}document\textgreater{}. Kedua tag document dan author
memiliki attribute yang sama yaitu name.

Author tag tidak memiliki tag penutup karena tidak memiliki elemen lain didalamnya, cara penulisannya
adalah \textless{}author \textfractionsolidus{}\textgreater{}, ini sama dengan menuliskan 
\textless{}author\textgreater{}\textless{}\textfractionsolidus{}author\textgreater{}.

Qt mendukung tiga cara untuk memanipulasi dokumen XML yaitu
QStreamReader, DOM, dan SAX.

Untuk menggunakan library XML pada Qt anda harus menambahkan library XML
pada Qt project file.

\begin{lstlisting}[language=c++]
#-------------------------------------------------
#
# Project created by QtCreator 2016-01-03T21:43:04
#
#-------------------------------------------------
QT += core
QT -= gui
QT += xml
TARGET = Contoh 5
CONFIG += console
CONFIG -= app_bundle
TEMPLATE = app
SOURCES += main.cpp
\end{lstlisting}

\section{DOM}\label{dom}

\index{DOM}DOM (Document Object Model) bekerja dengan cara merepresentasikan semua
dokumen XML menjadi bentuk tree yang mempunyai node dan menyimpanya
dalam memory.

\subsection{Membuat File XML}\label{membuat-file-xml}

Pertama kita akan mulai dengan membuat file \index{XML}XML menggunakan DOM, adapun
tahapan yang akan kita kerjakan adalah membuat node, menyambungkan node,
dan terakhir menulis dokumen XML. Agar lebih jelas mari kita coba buat
program dibawah ini.

\subsubsection*{Contoh  Membuat Nodes untuk membuat simple XML Document.}

\begin{enumerate}

\item
  Buka Qt Creator dan buat project Qt Console Application baru dengan
  nama contoh \ref{contoh12-5}, kemudian tulis kode berikut.

\begin{lstlisting}[language=c++, caption=Membuat Nodes untuk membuat simple XML Document, label=contoh12-5]
#include <QtCore/QCoreApplication>
#include <QDebug>
#include <QFile>
#include <QTextStream>
#include <QDomDocument>
#include <QDomElement>
#include <QDomText>
int main(int argc, char *argv[])
{
QCoreApplication a(argc, argv);
QDomDocument dokumen;
QDomElement mhs = dokumen.createElement("Mahasiswa");
mhs.setAttribute("Jurusan","TI");
QDomElement nim = dokumen.createElement("Nim");
QDomElement ipk = dokumen.createElement("Ipk");
QDomText nimtext = dokumen.createTextNode("22002321");
QDomText ipktext = dokumen.createTextNode("3.5");
dokumen.appendChild(mhs);
mhs.appendChild(nim);
nim.appendChild(nimtext);
mhs.appendChild(ipk);
ipk.appendChild(ipktext);
QFile file("simple.xml");
if(!file.open(QIODevice::WriteOnly | QIODevice::Text))
{
qDebug() << "File tidak ditemukan !";
a.exit(-1);
return a.exec();
}
QTextStream stream(&file);
stream << dokumen.toString();
qDebug() << "File XML berhasil dibuat ..";
file.close();
return a.exec();
}
\end{lstlisting}
\item
  Kemudian jalankan kode diatas dengan menekan tombol Ctrl+R, maka akan
  ditampilkan output sebagai berikut.
  
  \begin{lcverbatim}
  	content
  \end{lcverbatim}
\end{enumerate}

Hasil dari file XML ``simple.xml'' yang berhasil dibuat adalah :

 \begin{lstlisting}[language=xml]
 	<Mahasiswa Jurusan="TI">
 	<Nim>22002321</Nim>
 	<Ipk>3.5</Ipk>
 	</Mahasiswa>
 \end{lstlisting}

\textbf{Keterangan:}

\begin{itemize}

\item
  Untuk membuat dokumen XML menggunakan DOM, pertama buat objek
  QDomDocument .
\item
  Langkah selanjutnya adalah membuat objek QDomElement untuk membuat
  element Nim dan Ipk.
\item
  Untuk menambahkan text pada element tambahkan objek QDomText.
\item
  Setelah element dan text selesai dibuat, anda dapat memasangkan
  element dan teks menjadi child node dengan menggunakan method
  appendChild().
\item
  Setelah dokumen XML selesai dibuat, anda dapat menuliskan dokumen
  tersebut ke file teks dengan menggunakan objek QTextStream, pada
  contoh program diatas dokumen XML disimpan dalam file ``simple.xml''
\end{itemize}

\subsection{Membaca XML file}\label{membaca-xml-file}

Pada contoh sebelumnya anda telah bisa membuat dokumen XML dengan DOM,
sekarang kita akan mencoba membaca dokumen XML yang sebelumnya sudah
dibuat.

Pada contoh dibawah ini kita akan membaca dan memasukan file kedalam
QDomDocument dan kita juga akan mempelajari bagaimana cara menemukan
elemen dan teks yang terkandung dalam dokumen. Agar dokumen XML pada
file dapat diload kedalam QDomDocument maka dokumen XML tersebut harus
valid.

\subsubsection*{Contoh Membaca DOM dari dokumen XML.}

\begin{enumerate}

\item
  Untuk dokumen XML yang akan digunakan pada contoh program ini adalah
  dokumen XML yang sudah kita buat sebelumnya yaitu ``simple.xml''.
\item
  Buka Qt Creator dan buat project Qt Console Application baru dengan
  nama contoh \ref{contoh12-5}.
\item
  Jalankan dahulu program tersebut dengan menekan tombol Ctrl + R, agar
  folder simulator dengan nama Contoh build-\ref{contoh12-6}-Desktop\_Qt\_5\_5\_0\_MinGW\_32bit-Debug dibuat.
\item
  Kopikan file ``simple.xml'' yang akan dibaca kedalam folder contoh
  build-\ref{contoh12-6}-Desktop\_Qt\_5\_5\_0\_MinGW\_32bit-Debug.
\item
  Kemudian buka file main.cp, tulis kode untuk membaca file XML berikut:

\begin{lstlisting}[language=c++, caption=Contoh Membaca DOM dari dokumen XML, label=contoh12-6]
#include <QtCore/QCoreApplication>
#include <QFile>
#include <QTextStream>
#include <QDomDocument>
#include <QDomElement>
#include <QDomText>
#include <QDebug>
int main(int argc, char *argv[])
{
QCoreApplication a(argc, argv);
QFile file("simple.xml");
if(!file.open(QIODevice::ReadOnly | QIODevice::Text))
{
qDebug() << "File " << file.fileName() << " tidak ditemukan";
return a.exec();
}
QDomDocument dokumen;
if(!dokumen.setContent(&file))
{
qDebug() << "Gagal untuk parsing ke DOM tree";
file.close();
return a.exec();
}
QDomElement dokumenElemen = dokumen.documentElement();
QDomNode node = dokumenElemen.firstChild();
while(!node.isNull())
{
if(node.isElement())
{
QDomElement element = node.toElement();
qDebug() << "Element " << element.tagName();
qDebug() << "Atribut nama " << element.attribute("nama","tidak ada
attribute");
}
if(node.isText())
{
QDomText teks = node.toText();
qDebug() << teks.data();
}
node = node.nextSibling();
}
return a.exec();
}
\end{lstlisting}
\item
  Kemudian jalankan kode diatas dengan menekan tombol Ctrl+R, maka akan
  ditampilkan output sebagai berikut.
  
  \begin{lcverbatim}
  	content
  \end{lcverbatim}
\end{enumerate}

\textbf{Keterangan:}

\begin{itemize}

\item
  Untuk membaca data dari file, seperti biasa gunakan objek QFile dan
  jalankan method open().
\item
  Untuk mengambil data yang ada file untuk dimasukan kedalam objek
  QDomDocument gunakan method setContent().
\item
  Untuk mengakses data elemen pada QDomDocument, buat objek QDomElement.
\item
  Untuk mengambil node dari QDomElement, buat objek QDomNode. Method
  firstChild() digunakan untuk mengambil node awal pada QDomElement.
\item
  Kemudian jika node yang dibaca tidak null, cek apakah node tersebut
  berupa elemen, jika node tersebut element lakukan konversi ke objek
  QDomElement dan gunakan method tagName() untuk menampilkan nama
  element dan method attribute() untuk menampilkan atribut jika ada.
\item
  Cek juga apakah node bertipe teks dengan method isText(), jika teks
  konversi node kedalam objek QDomText kemudian tampilkan isi dari teks
  dengan menggunakan method toText().
\item
  Untuk berpindah ke node selanjutnya gunakan method nextSibling().
\end{itemize}

\subsection{Memodifikasi File XML}\label{memodifikasi-file-xml}

Pada topik sebelumnya kita sudah membahas bagaimana cara menulis dan
membaca data dari dokumen XML. Pada topik kali ini akan dibahas
bagaimana cara memodifikasi dokumen XML yang sudah anda load kedalam
objek QDomDocument.

\subsubsection*{Contoh  Modifikasi data dokumen XML.}

\begin{enumerate}


\item
  Buat project console aplication baru dengan nama contoh \ref{contoh12-7}, pilih Qt
  Simulator sebagai simulator yang digunakan. Jalankan program sehingga
  akan dibuat folder contoh build-\ref{contoh12-7}-Desktop\_Qt\_5\_5\_0\_MinGW\_32bit-Debug.
\item
  Pada folder contoh \ref{contoh12-7}-Desktop\_Qt\_5\_5\_0\_MinGW\_32bit-Debug buat file dengan nama
  ``simple.xml'', kemudian tulis dokumen XML berikut.

\begin{lstlisting}[language=xml]
<dokumen nama="Data Mahasiswa">
<Mahasiswa Jurusan="TI">
<Nim nama="Erick">22002321</Nim>
<Ipk>3.5</Ipk>
</Mahasiswa>
<Mahasiswa Jurusan="SI">
<Nim nama="Katon">23002333</Nim>
<Ipk>3.6</Ipk>
</Mahasiswa>
</dokumen>
\end{lstlisting}

\item
  Kemudian pada file main.cpp tambahkan kode berikut untuk melakukan
  modifikasi file xml yang sudah kita buat sebelumnya.

\begin{lstlisting}[language=c++, caption=Main program Modifikasi data dokumen XML, label=contoh12-7]
#include <QtCore/QCoreApplication>
#include <QDebug>
#include <QFile>
#include <QTextStream>
#include <QDomDocument>
#include <QDomElement>
#include <QDomText>
int main(int argc, char *argv[])
{
QCoreApplication a(argc, argv);
QFile fileAsli("simple.xml");
if(!fileAsli.open(QIODevice::ReadOnly | QIODevice::Text))
{
qDebug() << "File " << fileAsli.fileName() << " tidak ditemukan";
return a.exec();
}
QDomDocument dokumen;
if(!dokumen.setContent(&fileAsli))
{
qDebug() << "Gagal parsing file ke DOM tree";
fileAsli.close();
return a.exec();
}
fileAsli.close();
QDomElement elemenDokumen = dokumen.documentElement();
QDomNodeList elemen = elemenDokumen.elementsByTagName("Mahasiswa");
if(elemen.size() == 0)
{
QDomElement mhs = dokumen.createElement("Mahasiswa");
elemenDokumen.insertBefore(mhs,QDomNode());
}
else
{
QDomElement mhs = elemen.at(0).toElement();
QDomElement nama = dokumen.createElement("Nama");
QDomText textNama = dokumen.createTextNode("Erick Kurniawan");
nama.appendChild(textNama);
mhs.appendChild(nama);
}
QFile fileModif("simplemodif.xml");
if(!fileModif.open(QIODevice::WriteOnly | QIODevice::Text))
{
qDebug() << "Gagal untuk membaca file xml";
return a.exec();
}
QTextStream stream(&fileModif);
stream << dokumen.toString();
qDebug() << "File berhasil dimodifikasi dan disimpan pada file simplemodif.xml";
fileModif.close();
//membaca isi dari file simplemodif.xml
qDebug() << "Setelah dimodifikasi isi dari simplemodif.xml adalah";
if(!fileModif.open(QIODevice::ReadOnly | QIODevice::Text))
{
qDebug() << "Gagal membaca file xml";
return a.exec();
}
qDebug() << fileModif.readAll();
fileModif.close();
return a.exec();
}
\end{lstlisting}



\item
  Kemudian jalankan program diatas dengan menekan tombol Ctrl+R, maka
  akan ditampilkan output sebagai berikut.
  \begin{lcverbatim}
  	content
  \end{lcverbatim}
\item
  Untuk melihat hasil dari file xml yang sudah dimodifikasi, anda dapat
  membuka file ``simplemodif.xml'' yang ada pada folder contoh
  build-\ref{contoh12-7}-Desktop\_Qt\_5\_5\_0\_MinGW\_32bit-Debug.
\item
  Isi dari file ``simplemodif.xml'' adalah sebagai berikut

\begin{lstlisting}[language=xml]
<dokumen nama="Data Mahasiswa">
<Mahasiswa Jurusan="TI">
<Nim nama="Erick">22002321</Nim>
<Ipk>3.5</Ipk>
<Nama>Erick Kurniawan</Nama>
</Mahasiswa>
<Mahasiswa Jurusan="SI">
<Nim nama="Katon">23002333</Nim>
<Ipk>3.6</Ipk>
</Mahasiswa>
</dokumen>
\end{lstlisting}
\begin{lcverbatim}
	content
\end{lcverbatim}
\end{enumerate}

\textbf{Keterangan:}

\begin{itemize}

\item
  Langkah pertama baca file simple.xml yang akan dimodifikasi, kemudian
  masukan isinya kedalam QDomDocument.
\item
  Ambil root element dari dokumen menggunakan method documentElement()
  dan masukan kedalam objek bertipe QDomElement.
\item
  Kode QDomNodeList elemen =
  elemenDokumen.elementsByTagName(``Mahasiswa''); digunakan untuk
  mengambil semua element yg mempunyai tag \textless{}Mahasisw dan
  memasukannya kedalam variabel elemen yang bertipe QDomNodeList.
\item
  Kode if(elemen.size() == 0) digunakan untuk memeriksa apakah elemen
  dengan tag \textless{}Mahasisw ditemukan, jika tidak ditemukan maka
  buat elemen \textless{}Mahasisw, jika ditemukan ambil elemen
  \textless{}Mahasisw pertama kemudian tambahkan elemen baru
  \textless{}Nam beserta teks di dalam elemen \textless{}Mahasisw
  tersebut.
\item
  Langkah terakhir adalah membuat QTextStream dan menyimpan dokumen XML
  yang sudah dimodifikasi kedalam file simplemodif.xml.
\end{itemize}

\section{QXMLStream Reader}\label{qxmlstream-reader}

Selain menggunakan DOM untuk membaca dokumen XML, anda juga dapat
menggunakan objek QXMLStreamReader. QXMLStreamReader adalah class parser
XML tercepat dan termudah untuk digunakan, karena QXMLStreamReader
parser bekerja secara incremental sehingga mempermudah pembacaan tag.
QXMLStreamReader juga cocok digunakan untuk membaca file yang berukuran
besar yang tidak cocok jika disimpan di memory. Beberapa token yang
digunakan adalah seperti tabel \ref{tabel12-1}

\begin{tabular}{|l|l| p{5cm} |l|}
	\hline 
	\textbf{Token Type}  & \textbf{Contoh} & \textbf{Getter Fuctions }\\ 	\hline 
	StartDocument &	N/A &	isStandaloneDocument() \\ \hline
	EndDocument & 	N/A &	isStandaloneDocument() \\ \hline
	StartElement &<item> &	namespaceUri() ,  name() ,  attributes() ,	namespaceDeclarations()\\ \hline
	EndElement &</item>&	namespaceUri() ,  name()\\ \hline
	Characters& AT\&amp;T&	text() ,  isWhitespace() ,  isCDATA() \\ \hline
	Comment &<!-- fix --	>&	text()\\ \hline
	DTD &<!DOCTYPE \dots>&	text() ,  notationDeclarations() ,	entityDeclarations()\\ \hline
	EntityReference &\&trade;&	name() ,  text()\\ \hline
	ProcessingInstruction & <?alert?>&	processingInstructionTarget() ,processingInstructionData()\\ \hline
	Invalid &>\&<!&	error() ,  errorString()\\ \hline
\end{tabular} 

Misal anda memiliki dokumen XML sebagai berikut

\begin{lstlisting}[language=xml]
<doc>
<quote>Einmal ist keinmal</quote>
</doc>
\end{lstlisting}

Setiap anda menggunakan method readNext() maka akan dibaca satu token. Untuk kode diatas kita
dapat membaca pertoken.

\begin{lstlisting}[language=c++, numbers=none]
StartDocument
StartElement (name() == "doc")
StartElement (name() == "quote")
Characters (text() == "Einmal ist keinmal")
EndElement (name() == "quote")
EndElement (name() == "doc")
EndDocument
\end{lstlisting}

Setelah memanggil method readNext(), anda dapat mengecek token yang
sedang aktif dengan menggunakan isStartElement(), isCharacter(), atau
menggunakan fungsi yang mirip seperti state().

\subsubsection*{Contoh  Menggunakan QXMLStream Reader untuk membaca XML.}

\begin{enumerate}
\item
  Buat aplikasi console baru dengan nama contoh \ref{contoh12-7}, pilih Qt Simulator
  untuk menampilkan outputnya.
\item
  Buat file simple.xml dan buat dokumen XML sebagai berikut

\begin{lstlisting}[language=xml]
<Mahasiswas>
<Mahasiswa>
<Nim nama="Erick">22002321</Nim>
<Ipk>3.5</Ipk>
</Mahasiswa>
<Mahasiswa>
<Nim nama="Katon">23002333</Nim>
<Ipk>3.6</Ipk>
</Mahasiswa>
</Mahasiswas>
\end{lstlisting}
\item
  Kemudian ketikan kode berikut pada main.cpp untuk membaca data dari
  dokumen XML.

\begin{lstlisting}[language=c++, caption=Menggunakan QXMLStream Reader untuk membaca XML, label=contoh12-8]
#include <QtCore/QCoreApplication>
#include <QDebug>
#include <QFile>
#include <QXmlStreamReader>
int main(int argc, char *argv[])
{
QCoreApplication a(argc, argv);
QFile file("simple.xml");
if(!file.open(QIODevice::ReadOnly | QIODevice::Text))
{
qDebug() << "File tidak ditemukan ";
return a.exec();
}
QXmlStreamReader reader;
reader.setDevice(&file);
while(!reader.atEnd())
{
reader.readNext();
if(reader.isStartElement())
{
qDebug() << reader.name();
if(reader.name() == "Nim")
{
qDebug() << "Nama Attribute :" << reader.attributes().value("nama");
qDebug() << "Teks : " << reader.readElementText();
}
}
}
file.close();
return a.exec();
}
\end{lstlisting}
\item
  Jalankan aplikasi tersebut, maka akan ditampikan output sebagai
  berikut.
  \begin{lcverbatim}
  	content
  \end{lcverbatim}
\end{enumerate}

\textbf{Keterangan:}

\begin{itemize}

\item
  Untuk membaca dokumen XML menggunakan QXMLStreamReader anda dapat
  melakukan looping dengan memeriksa apakah sudah sampai pada akhir
  dokumen while(!reader.atEnd()).
\item
  Method reader.readNext(); digunakan untuk berpindah token.
\item
  Method reader.isStartElement() digunakan untuk mengecek apakah token
  yg sekarang aktif adalah elemen awal.
\item
  Kode if(reader.name() == ``Nim'') digunakan untuk mengecek apakah nama
  elemen adalah ``Nim'' jika ya baca attribute dan teks pada elemen
  tersebut untuk ditampilkan.
\end{itemize}

\subsubsection*{Contoh  Membuat dokumen XML dengan QXMLStreamWriter.}

\begin{enumerate}

\item
  Buat aplikasi console dengan nama contoh \ref{contoh12-8}, kemudian tulis kode
  berikut

\begin{lstlisting}[language=c++, caption=Membuat dokumen XML dengan QXMLStreamWriter, label=contoh12-9]
#include <QtCore/QCoreApplication>
#include <QDebug>
#include <QFile>
#include <QXmlStreamWriter>
int main(int argc, char *argv[])
{
QCoreApplication a(argc, argv);
QFile file("simple.xml");
if(!file.open(QIODevice::WriteOnly | QIODevice::Text))
{
qDebug() << "File tidak ditemukan..";
return a.exec();
}
QXmlStreamWriter writer(&file);
writer.setAutoFormatting(true);
writer.writeStartDocument();
writer.writeStartElement("Books");
writer.writeStartElement("Book");
writer.writeStartElement("Author");
writer.writeAttribute("Name","Erick Kurniawan");
writer.writeAttribute("Title","Qt Programming");
writer.writeEndDocument();
file.close();
qDebug() << "File sudah berhasil dibuat !";
return a.exec();
}
\end{lstlisting}
\item
  Jalankan program diatas untuk menggenerate dokumen XML dengan nama
  simple.xml.
  \begin{lcverbatim}
File sudah berhasil di buat !
  \end{lcverbatim}
\item
  Isi dari dokumen XML yang barusan anda buat adalah sebagai berikut.

\begin{lstlisting}[language=xml]
<Books>
<Book>
<Author name="Erick Kurniawan" Title="ASP.NET 3.5"/>
</Book>
</Books>
\end{lstlisting}

\end{enumerate}

\textbf{Keterangan:}

\begin{itemize}

\item
  Anda dapat menggunakan QXMLStreamWriter untuk membuat dokumen XML.
\item
  Kode writer.setAutoFormatting(true); digunakan untuk memformat secara
  otomatis dokumen XML yang akan dibuat, misal menambahkan line break
  dan indentation pada bagian yang kosong pada elemen. Tujuan utamanya
  adalah memisahkan data menjadi beberapa baris sehingga membantu dalam
  pembacaan dokumen.
\item
  Kode writer.writeStartDocument(); digunakan pada saat pertama kali
  dokumen akan dibuat.
\item
  Kode writer.writeStartElement(``Books''); digunakan untuk menuliskan
  elemen Books.
\item
  Kode writer.writeAttribute(``Name'',``Erick Kurniawan''); digunakan
  untuk menambahkan attribute pada elemen tertentu.
\item
  Kode writer.writeEndDocument(); digunakan untuk menutup semua start
  element yang sebelumnya dibuat dengan method writeStartElement().
\end{itemize}
