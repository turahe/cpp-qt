\textbf{📋 Apa yang akan dipelajari}

Pada bab ini kita akan mempelajari tentang pemrograman GUI (Graphical User Interface) dengan Qt:

\begin{itemize}
\item Pengenalan GUI dan Qt Widgets
\item Signal dan Slot mechanism
\item Membuat aplikasi GUI sederhana
\item Layout management
\item Event handling
\end{itemize}

\minitoc

\section{🎨 Pengenalan GUI (Graphical User Interface)}

\subsection{Apa itu GUI?}

GUI (Graphical User Interface) adalah antarmuka pengguna yang menggunakan elemen grafis seperti ikon, tombol, menu, dan jendela untuk berinteraksi dengan pengguna. GUI membuat aplikasi lebih user-friendly dibandingkan dengan aplikasi berbasis teks (command line).

\subsection{Keunggulan GUI}

\begin{itemize}
\item \textbf{User-friendly} - Mudah digunakan bahkan untuk pemula
\item \textbf{Visual} - Informasi ditampilkan secara grafis
\item \textbf{Intuitif} - Pengguna dapat langsung memahami fungsi
\item \textbf{Modern} - Sesuai dengan standar aplikasi masa kini
\end{itemize}

\subsection{Qt sebagai Framework GUI}

Qt adalah framework C++ yang powerful untuk membuat aplikasi GUI cross-platform. Qt menyediakan:
\begin{itemize}
\item \textbf{QtWidgets} - Widget tradisional untuk desktop
\item \textbf{QtQuick} - Framework modern untuk UI yang dinamis
\item \textbf{Qt Designer} - Tool visual untuk membuat UI
\item \textbf{Signal-Slot} - Mekanisme komunikasi antar komponen
\end{itemize}

\section{🏗️ Komponen Dasar Qt GUI}

\subsection{QWidget - Base Class}

QWidget adalah kelas dasar untuk semua elemen GUI dalam Qt. Setiap widget dapat:
\begin{itemize}
\item Memiliki parent widget
\item Menampilkan konten
\item Merespons event
\item Memiliki layout
\end{itemize}

\subsection{Jenis-jenis Widget}

\subsubsection{Widget Input}
\begin{itemize}
\item \textbf{QPushButton} - Tombol yang dapat diklik
\item \textbf{QLineEdit} - Kotak input teks satu baris
\item \textbf{QTextEdit} - Area input teks multi-baris
\item \textbf{QComboBox} - Dropdown list
\item \textbf{QCheckBox} - Checkbox untuk pilihan
\item \textbf{QRadioButton} - Radio button untuk pilihan tunggal
\end{itemize}

\subsubsection{Widget Display}
\begin{itemize}
\item \textbf{QLabel} - Menampilkan teks atau gambar
\item \textbf{QTextBrowser} - Menampilkan teks dengan format
\item \textbf{QProgressBar} - Bar progress
\item \textbf{QSlider} - Slider untuk nilai numerik
\end{itemize}

\subsubsection{Container Widget}
\begin{itemize}
\item \textbf{QGroupBox} - Mengelompokkan widget
\item \textbf{QTabWidget} - Tab untuk multiple pages
\item \textbf{QScrollArea} - Area dengan scroll
\item \textbf{QFrame} - Frame untuk border dan styling
\end{itemize}

\section{🔗 Signal dan Slot}

\subsection{Konsep Signal dan Slot}

Signal dan Slot adalah mekanisme komunikasi antar objek dalam Qt yang memungkinkan:
\begin{itemize}
\item \textbf{Signal} - Event yang dikirim oleh widget
\item \textbf{Slot} - Fungsi yang merespons signal
\item \textbf{Connection} - Menghubungkan signal dengan slot
\end{itemize}

\subsection{Cara Kerja Signal-Slot}

\begin{enumerate}
\item Widget mengirim signal ketika event terjadi
\item Signal terhubung ke slot melalui connection
\item Slot dijalankan untuk merespons event
\item Program dapat merespons interaksi pengguna
\end{enumerate}

\subsection{Sintaks Signal-Slot}

\lstinputlisting[language=c++]{../code/14-gui-code-1.c++}

\section{💻 Membuat Aplikasi GUI Sederhana}

\subsection{Struktur Program GUI}

Program GUI Qt terdiri dari:
\begin{itemize}
\item \textbf{Header file} (.h) - Deklarasi kelas dan slot
\item \textbf{Source file} (.cpp) - Implementasi kelas dan slot
\item \textbf{UI file} (.ui) - Desain antarmuka (opsional)
\item \textbf{Main file} - Entry point aplikasi
\end{itemize}

\subsection{Contoh Program GUI Sederhana}

\subsubsection{Header File (mainwindow.h)}

\lstinputlisting[language=c++, caption=Header file untuk aplikasi GUI, label=gui-header]{../code/gui-mainwindow.h}

\subsubsection{Source File (mainwindow.cpp)}

\begin{lstlisting}[language=c++, caption=Source file untuk aplikasi GUI, label=gui-source]{../code/gui-mainwindow.cpp}
#include "mainwindow.h"
#include <QVBoxLayout>
#include <QHBoxLayout>
#include <QWidget>

MainWindow::MainWindow(QWidget *parent)
    : QMainWindow(parent)
{
    // Set window properties
    setWindowTitle("Aplikasi GUI Sederhana");
    setFixedSize(400, 300);

    // Create central widget
    QWidget *centralWidget = new QWidget(this);
    setCentralWidget(centralWidget);

    // Create layout
    QVBoxLayout *mainLayout = new QVBoxLayout(centralWidget);

    // Create buttons
    buttonA = new QPushButton("Tampilkan Gambar A", this);
    buttonB = new QPushButton("Tampilkan Gambar B", this);

    // Create image label
    imageLabel = new QLabel("Klik tombol untuk menampilkan gambar", this);
    imageLabel->setAlignment(Qt::AlignCenter);
    imageLabel->setStyleSheet("QLabel { border: 2px solid gray; padding: 10px; }");

    // Add widgets to layout
    mainLayout->addWidget(buttonA);
    mainLayout->addWidget(buttonB);
    mainLayout->addWidget(imageLabel);

    // Connect signals to slots
    connect(buttonA, SIGNAL(clicked()), this, SLOT(onButtonAClicked()));
    connect(buttonB, SIGNAL(clicked()), this, SLOT(onButtonBClicked()));
}

MainWindow::~MainWindow()
{
}

void MainWindow::onButtonAClicked()
{
    imageLabel->setText("Gambar A ditampilkan");
    imageLabel->setStyleSheet("QLabel { border: 2px solid blue; padding: 10px; background-color: lightblue; }");
}

void MainWindow::onButtonBClicked()
{
    imageLabel->setText("Gambar B ditampilkan");
    imageLabel->setStyleSheet("QLabel { border: 2px solid red; padding: 10px; background-color: lightcoral; }");
}
\end{lstlisting}

\subsubsection{Main File (main.cpp)}

\begin{lstlisting}[language=c++, caption=Main file untuk aplikasi GUI, label=gui-main]{../code/gui-main.cpp}
#include <QApplication>
#include "mainwindow.h"

int main(int argc, char *argv[])
{
    QApplication app(argc, argv);
    
    MainWindow window;
    window.show();
    
    return app.exec();
}
\end{lstlisting}

\section{📐 Layout Management}

\subsection{Pentingnya Layout}

Layout management memastikan:
\begin{itemize}
\item Widget tersusun rapi dan teratur
\item Aplikasi responsive terhadap perubahan ukuran
\item UI konsisten di berbagai platform
\item Mudah untuk maintenance dan modifikasi
\end{itemize}

\subsection{Jenis-jenis Layout}

\subsubsection{QVBoxLayout (Vertical Layout)}
\begin{itemize}
\item Menyusun widget secara vertikal
\item Widget ditumpuk dari atas ke bawah
\item Cocok untuk form atau menu vertikal
\end{itemize}

\subsubsection{QHBoxLayout (Horizontal Layout)}
\begin{itemize}
\item Menyusun widget secara horizontal
\item Widget disusun dari kiri ke kanan
\item Cocok untuk toolbar atau status bar
\end{itemize}

\subsubsection{QGridLayout (Grid Layout)}
\begin{itemize}
\item Menyusun widget dalam bentuk grid/tabel
\item Widget dapat menempati multiple cell
\item Cocok untuk form yang kompleks
\end{itemize}

\subsubsection{QFormLayout (Form Layout)}
\begin{itemize}
\item Khusus untuk form input
\item Label dan input field otomatis tersusun
\item Cocok untuk dialog atau form pengaturan
\end{itemize}

\section{🎯 Event Handling}

\subsection{Konsep Event}

Event adalah kejadian yang terjadi dalam aplikasi GUI:
\begin{itemize}
\item \textbf{Mouse events} - klik, drag, scroll
\item \textbf{Keyboard events} - tekan tombol, release
\item \textbf{Window events} - resize, close, focus
\item \textbf{Custom events} - event yang dibuat sendiri
\end{itemize}

\subsection{Cara Menangani Event}

\subsubsection{Menggunakan Signal-Slot}
\lstinputlisting[language=c++]{../code/14-gui-code-2.c++}

\subsubsection{Override Event Handler}
\lstinputlisting[language=c++]{../code/14-gui-code-3.c++}

\section{🎨 Styling dan Theming}

\subsection{Qt Style Sheets}

Qt Style Sheets (QSS) memungkinkan styling widget menggunakan CSS-like syntax:
\begin{itemize}
\item \textbf{Selector} - Memilih widget yang akan di-style
\item \textbf{Property} - Properti visual (warna, font, border)
\item \textbf{Value} - Nilai dari properti
\end{itemize}

\subsection{Contoh Style Sheet}

\lstinputlisting[language=c++]{../code/14-gui-code-4.c++}

\section{🔧 Best Practices GUI Programming}

\subsection{Tips Menulis GUI yang Baik}

\begin{enumerate}
\item \textbf{Gunakan layout manager} - Jangan set posisi widget secara manual
\item \textbf{Berikan nama yang jelas} - Nama widget dan fungsi yang deskriptif
\item \textbf{Gunakan signal-slot} - Hindari polling atau callback manual
\item \textbf{Handle error dengan baik} - Validasi input dan error handling
\item \textbf{Test di berbagai platform} - Pastikan konsisten di Windows, Mac, Linux
\end{enumerate}

\subsection{Kesalahan Umum}

\begin{itemize}
\item \textbf{Lupa connect signal-slot} - Widget tidak merespons event
\item \textbf{Manual positioning} - UI tidak responsive
\item \textbf{Memory leak} - Lupa delete widget yang dibuat
\item \textbf{Blocking UI thread} - Aplikasi freeze saat operasi berat
\end{itemize}

\section{📱 Membuat Aplikasi Modern}

\subsection{Qt Quick vs Qt Widgets}

\begin{center}
\begin{tabular}{|l|l|l|}
\hline
\textbf{Aspek} & \textbf{Qt Widgets} & \textbf{Qt Quick} \\
\hline
Target platform & Desktop & Desktop, Mobile, Embedded \\
\hline
Performance & C++ native & QML + JavaScript \\
\hline
Learning curve & Moderate & Steep (QML) \\
\hline
Animation & Limited & Rich \\
\hline
Touch support & Basic & Excellent \\
\hline
\end{tabular}
\end{center}

\subsection{QML untuk UI Modern}

QML (Qt Modeling Language) adalah bahasa deklaratif untuk membuat UI modern:
\begin{itemize}
\item \textbf{Declarative} - UI dideklarasikan, bukan diprogram
\item \textbf{Animation} - Animasi yang smooth dan natural
\item \textbf{Responsive} - UI yang responsive terhadap input
\item \textbf{Cross-platform} - Sama di semua platform
\end{itemize}

\section{🔍 Latihan dan Project}

\subsection{📝 Latihan 1: Kalkulator Sederhana}

Buat aplikasi kalkulator dengan GUI yang memiliki:
\begin{itemize}
\item Display untuk menampilkan angka
\item Tombol angka 0-9
\item Tombol operator (+, -, *, /)
\item Tombol equals (=) dan clear (C)
\end{itemize}

\subsection{📝 Latihan 2: Text Editor Sederhana}

Buat text editor dengan fitur:
\begin{itemize}
\item Area text untuk menulis
\item Menu File (New, Open, Save)
\item Menu Edit (Copy, Paste, Cut)
\item Status bar untuk informasi
\end{itemize}

\subsection{📝 Latihan 3: Image Viewer}

Buat image viewer dengan:
\begin{itemize}
\item Area untuk menampilkan gambar
\item Tombol untuk load image
\item Tombol untuk zoom in/out
\item Tombol untuk rotate image
\end{itemize}

\section{📚 Referensi dan Bacaan Lanjutan}

Qt GUI programming adalah topik yang luas dan terus berkembang. Untuk pemahaman yang lebih mendalam, pembaca dapat merujuk pada:

\begin{itemize}
\item \textbf{Qt Documentation} - Dokumentasi resmi Qt\footnote{Qt Company. (2023). "Qt Documentation". https://doc.qt.io/}
\item \textbf{C++ GUI Programming with Qt 4} oleh Jasmin Blanchette\footnote{Blanchette, J., \& Summerfield, M. (2006). "C++ GUI Programming with Qt 4". Prentice Hall.}
\item \textbf{Qt 5 C++ GUI Programming Cookbook} oleh Lee Zhi Eng\footnote{Eng, L. Z. (2016). "Qt 5 C++ GUI Programming Cookbook". Packt Publishing.}
\item \textbf{Qt Quick} - Framework modern untuk UI\footnote{Qt Company. (2023). "Qt Quick". https://doc.qt.io/qt-6/qtquick-index.html}
\end{itemize}

\section{🎉 Kesimpulan}

Pemrograman GUI dengan Qt membuka dunia kemungkinan untuk membuat aplikasi desktop yang powerful dan user-friendly. Dengan memahami konsep widget, signal-slot, layout management, dan event handling, Anda dapat membuat aplikasi GUI yang profesional.

\begin{center}
\textbf{Selamat! Anda telah menguasai dasar-dasar GUI Programming dengan Qt} 🎨
\end{center}

\vspace{1cm}

\begin{center}
\textit{--- Bab selanjutnya: File, Stream, dan XML ---}
\end{center}

