\chapter*{Prakata}\label{prakata}

Bahasa pemrograman C/C++ merupakan bahasa yang popular didalam
pengajaran pada computer sains maupun pada kalangan programmer yang
mengembangkan system software maupun aplikasi.

Bahasa C/C++ sifatnya portable, karena compilernya tersedia hampir pada
semua arsitektur computer maupun system operasi, sehingga investasi
waktu dan tenaga yang anda lakukan dalam mempelajari bahasa pemrograman
ini memiliki nilai strategis yang sangat menjanjikan.

Bahasa C/C++ merupakan bahasa yang sangat ketat dalam pemakaian type
data maupun penulisannya yang case sensitif, hal ini berarti programmer
di tuntut kedisiplinannya dalam penulisan program.

Sesuatu fasilitas yang tersedia dalam C/C++ yang tidak ditemukan pada
bahasa pemrogaman lainnya adalah pointer, dengan pemanfaatan pointer
programmer dapat melakukan manipulasi memori secara langsung.

Dewasa ini beberapa bahasa yang memiliki syntax penulisan yang
menyerupai C/C++ adalah Java, Javascript dan PHP, yang artinya bahwa
kemampuan pemrograman dengan C/C++ akan mempermudah anda untuk
mempelajari bahasa modern seperti Java maupun C\# (dibaca C sharp).

Akhirnya penulis mengucapkan selamat belajar dan semoga buku ini dapat
memberi manfaat yang sebesarnya dalam pembelajaran mata kuliah C/C++
Programming.
