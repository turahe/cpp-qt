\section*{Tentang Penulis}

\subsection*{Nur Wachid}

\textbf{Head of Software Engineering} \\
PT. Lingkar Kreasi Teknologi \\
Tangerang, Indonesia

\vspace{0.5cm}

\begin{quote}
\textit{Experienced programmer with a strong background in software development and a passion for problem-solving. Proficient in designing, coding, testing, and maintaining software applications. Adept at collaborating with cross-functional teams to deliver high-quality solutions. Specialized in web development and mobile app development.}
\end{quote}

\subsection*{Profil Profesional}

Nur Wachid adalah seorang profesional di bidang software engineering dengan pengalaman lebih dari 7 tahun dalam pengembangan aplikasi. Saat ini menjabat sebagai \textbf{Head of Software Engineering} di PT. Lingkar Kreasi Teknologi, memimpin tim pengembang dalam menciptakan solusi teknologi yang inovatif.

\subsection*{Keahlian Utama}

\begin{itemize}
\item \textbf{Software Development} - Desain, pengembangan, testing, dan maintenance aplikasi
\item \textbf{Web Development} - Pengembangan aplikasi web modern dan responsif
\item \textbf{Mobile App Development} - Pengembangan aplikasi mobile cross-platform
\item \textbf{Enterprise Resource Planning (ERP)} - Sistem manajemen perusahaan
\item \textbf{Data Pipelines} - Otomatisasi dan pengelolaan data
\item \textbf{Machine Learning} - Implementasi AI dan machine learning
\item \textbf{DevOps Engineering} - Infrastructure as Code, CI/CD, Containerization
\item \textbf{Backend Development} - RESTful API, Database optimization, Security
\end{itemize}

\subsection*{Pengalaman Profesional}

\subsubsection*{Head of Software Engineering @ PT. Lingkar Kreasi Teknologi}
\textit{Agustus 2022 - Desember 2024 | Tangerang, Indonesia}

\begin{itemize}
\item Memimpin tim pengembang dengan meningkatkan produktivitas tim sebesar 20\%
\item Mengelola proyek kompleks dengan tingkat kepuasan klien 95\%
\item Merancang arsitektur inovatif yang mengurangi biaya server 30\% dengan peningkatan traffic 200\%
\item Membimbing developer junior dan meningkatkan keterampilan teknis tim
\item Menerapkan proses code review yang mengurangi bug kritis sebesar 40\%
\end{itemize}

\subsubsection*{VP DevOps Engineer @ PT. Lingkar Kreasi}
\textit{November 2019 - Agustus 2022 | Bandung, Indonesia}

\begin{itemize}
\item Merancang pipeline deployment otomatis yang mengurangi waktu deployment dari hari menjadi menit
\item Menerapkan Infrastructure as Code menggunakan Terraform dan AWS CloudFormation
\item Mengarsitektur infrastruktur high availability dengan uptime 99.99\%
\item Mengelola container menggunakan Kubernetes dan Docker Swarm
\item Menetapkan CI/CD pipeline yang mengurangi kegagalan deployment 50\%
\end{itemize}

\subsubsection*{Backend Developer @ PT. Rakhasa Artha Wisesa}
\textit{Februari 2021 - November 2022 | Jakarta, Indonesia}

\begin{itemize}
\item Meningkatkan performa database query sebesar 40\%
\item Menerapkan keamanan tingkat lanjut dengan zero security breach dalam 2 tahun
\item Mengembangkan RESTful API yang mengurangi response time 50\%
\item Mengintegrasikan payment gateway yang meningkatkan transaksi berhasil 20\%
\end{itemize}

\subsubsection*{Project Lead Developer @ PT. Lingkar Kreasi}
\textit{November 2018 - November 2019 | Bandung, Indonesia}

\begin{itemize}
\item Memimpin tim cross-functional dengan tingkat penyelesaian proyek tepat waktu 95\%
\item Mempertahankan tingkat kepuasan klien 90\% atau lebih tinggi
\item Mengidentifikasi dan mitigasi risiko proyek secara proaktif
\item Mengelola anggaran proyek dengan efisiensi 15\%
\end{itemize}

\subsubsection*{Web Developer @ PT Danadipa Central Niaga}
\textit{November 2017 - November 2018 | Yogyakarta, Indonesia}

\begin{itemize}
\item Mengembangkan dan meluncurkan website skala besar untuk klien
\item Mengoptimalkan performa website untuk meningkatkan user experience
\item Menerapkan keamanan website dengan enkripsi data dan proteksi cyberattack
\end{itemize}

\subsection*{Fokus Saat Ini}

Saat ini Nur Wachid fokus pada pengembangan:
\begin{itemize}
\item \textbf{Enterprise Resource Planning (ERP)} - Sistem manajemen perusahaan terintegrasi
\item \textbf{Data Pipelines} - Otomatisasi pengelolaan dan analisis data
\item \textbf{Machine Learning} - Implementasi AI untuk optimasi bisnis
\item \textbf{Open Source} - Kontribusi aktif dalam proyek open source
\end{itemize}

\subsection*{Kontribusi Pendidikan}

Nur Wachid aktif menulis buku dalam bidang:
\begin{itemize}
\item \textbf{Pemrograman} - Buku panduan C++ dan Qt untuk pemula
\item \textbf{Data Science} - Analisis data dan machine learning
\item \textbf{Teknik Elektro} - Bidang keahlian akademis
\end{itemize}

\subsection*{Filosofi Pengembangan}

\begin{quote}
\textit{"I am always looking to learn new things. I am a strong advocate for open source and I am always interested in working on new projects with new people."}
\end{quote}

\subsection*{Kontak}

\begin{itemize}
\item \textbf{Email}: [Email tersedia di profil]
\item \textbf{GitHub}: [GitHub tersedia di profil]
\item \textbf{LinkedIn}: [LinkedIn tersedia di profil]
\item \textbf{Website}: \url{https://www.wach.id}
\item \textbf{WhatsApp}: [WhatsApp tersedia di profil]
\end{itemize}

\vspace{1cm}

\begin{center}
\textit{--- Penulis buku "Belajar C++ dengan Qt Creator" ---}
\end{center}