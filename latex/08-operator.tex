\textbf{Agenda}

Pada chapter ini kita akan membahas beberapa topik tentang penggunaan
Operator Types dan Operator Overloading, adapun topik yang akan dibahas
adalah

\minitoc

\section{Operator pada C++}\label{operator-pada-cpp}

Pada bab awal kita sudah mempelajari berbagai macam operator (+, -, /,
\textgreater{}, \textless{}) yang dapat digunakan pada tipe data yang
sudah ada di C++ seperti \texttt{int}, \texttt{float}, \texttt{bool},
dll. Namun jika anda ingin menggunakan operator tersebut pada tipe data
yang anda definisikan sendiri seperti tipe data Class, maka anda dapat
menggunakan keyword operator .

\begin{lstlisting}[language=c++, numbers=none]
return_type operator operator_symbol (...parameter list...);
\end{lstlisting}

Penggunaan keyword operator sebenarnya mirip dengan penggunaan fungsi ,
hanya anda dapat menggunakan operator symbol seperti (+, -,
\textgreater{},\textless{} , =, dll). Mungkin anda bertanya kenapa harus
menggunakan keyword operator jika anda dapat menggunakan fungsi ?
Ilustrasi dibawah ini akan menunjukan kenapa kita membutuhkan operator.

\begin{lstlisting}[language=c++, numbers=none]
CKataString strKata1("Hello");
CKataString strKata2("World");
\end{lstlisting}

Jika anda menginginkan untuk menggabungkan kedua kata tersebut anda
dapat membuat function \texttt{Concatenate} seperti berikut:

\begin{lstlisting}[language=c++, numbers=none]
CKataString strGabung;
strGabung = strKata1.Concatenate(strKata2);
\end{lstlisting}

Selain cara seperti diatas akan lebih natural jika anda menulis kode
sebagai berikut:

\begin{lstlisting}[language=c++, numbers=none]
CKataString strGabung;
strGabung = strKata1 + strKata2;
\end{lstlisting}

Walaupun hasil dari kedua cara penulisan diatas sama, namun penggunaan
operator + untuk menggabungkan string akan lebih intuitif dan mudah
dipahami.

Ada 2 macam operator yang terdapat di C++ yaitu unary dan binary.

\subsection{Unary Operator}\label{unary-operator}

Unary operator hanya mempunyai single operand saja, cara penulisan unary
operator adalah sebagai berikut.

\begin{lstlisting}[language=c++, numbers=none]
return_type operator operator_type (parameter_type)
{
// ... implementation
}
\end{lstlisting}

Tipe dari unary operator yang dapat digunakan adalah

\begin{longtable}[]{@{}ll@{}}
\toprule
Operator & Name\tabularnewline
\midrule
\endhead
++ & Increment\tabularnewline
-- & Decerement\tabularnewline
* & Pointer Dereference\tabularnewline
-\textgreater{} & Member Selection\tabularnewline
! & Logika NOT\tabularnewline
\& & Address-of\tabularnewline
\textasciitilde{} & One's Complement\tabularnewline
+ & Unary Plus\tabularnewline
- & Unary Negation\tabularnewline
\bottomrule
\end{longtable}

Pada Labs1 akan ditunjukan penggunaan operator increment. Pada contoh
dibawah ini kita akan membuat Class Kalender yang mempunyai tiga class
member yang merepresentasikan hari, bulan, dan tahun (tipe integer),
anda dapat menggunakan operator ++ untuk menambahkan hari.

\subsubsection*{Contoh  Menggunakan Increment Operator (Notasi prefix).}

\begin{enumerate}

\item
  Buka Qt Creator dan buat project Qt Console Application baru dengan
  nama contoh \ref{contoh8-1}, kemudian tulis kode berikut.

\lstinputlisting[language=c++, caption=Menggunakan Increment Operator (Notasi prefix), label=contoh8-1]{../code/contoh8-1.cpp}


\item
  Kemudian jalankan kode diatas dengan menekan tombol Ctrl+R, outputnya
  adalah sebagai berikut.
\end{enumerate}

\begin{lcverbatim}
Membuat object kalender dan memberi inisialisasi
23 / 10 / 2010

Tanggal setelah notasi prefik dijalankan
24 / 10 / 2010
\end{lcverbatim}

\subsubsection*{Keterangan}

\begin{itemize}

\item
  Class Kalender berisi tiga member variabel yaitu \texttt{\_hari},
  \texttt{\_bulan}, dan \texttt{\_tahun} yang merepresentasikan waktu
  tertentu.
\item
  operator \texttt{++} digunakan untuk menambahkan 1 hari kedalam objek
  Kalender, dengan menggunakan operator \texttt{++} penulisan menjadi
  lebih intuitif dan mudah dipahami, misal untuk menambahkan 1 hari
  kedalam objek Kalender anda dapat menuliskan \texttt{++objKal}.
\item
  Untuk memanggil operator \texttt{++} pada program diatas digunakan
  notasi prefix (tanda ++ dituliskan sebelum nama objek).
\item
  Karena kode diatas menggunakan notasi prefix maka pada operator ++
  akan mengakses objek by reference.
\end{itemize}

Ada perbedaan penulisan notasi prefix dan postfix, sebagai contoh anda
dapat melihat kode dibawah ini.

\begin{lstlisting}[language=c++, numbers=none]
int bil1 = 22;
int bil2 = bil1++;
//mengcopy nilai lama dari bil1
cout << "bil2 : " << bil2;
//nilai bil1 setelah di increment
cout << "bil1 : " << bil1;
\end{lstlisting}

Nilai \texttt{bil2} adalah \texttt{22}, karena yang dimasukan kedalam
\texttt{bil2} adalah nilai lama dari \texttt{bil1}, baru setelah itu
\texttt{bil1} di increment.

Untuk contoh \ref{contoh8-2} dibawah ini kita akan mencoba menggunakan notasi
\texttt{postfix}, dengan notasi \texttt{postfix} yang dilakukan adalah
menduplikat objek yang diinputkan, melakukan increment dan mengembalikan
objek tersebut by value.

\subsubsection*{Contoh  Menggunakan Operator Increment (notasi postfix).}

\begin{enumerate}

\item
  Buka Qt Creator dan buat project Qt Console Application baru dengan
  nama contoh \ref{contoh8-2}, kemudian tulis kode berikut.

\lstinputlisting[language=c++, caption=Menggunakan Operator Increment (notasi postfix), label=contoh8-2]{../code/contoh8-2.cpp}

\item
  Kemudian jalankan kode diatas dengan menekan tombol Ctrl+R, outputnya
  adalah sebagai berikut.
\end{enumerate}

\begin{lcverbatim}
Membuat object kalender dan memberi inisialisasi
23 / 10 / 2010

Menggunakan notasi postfix
objLama : 23 / 10 / 2010
objKal : 24 / 10 / 2010
\end{lcverbatim}

\subsubsection*{Keterangan}

\begin{itemize}

\item
  Output yang dihasilkan akan sama dengan kode sebelumnya, hanya saja
  penulisan operator \texttt{++} menggunakan notasi \texttt{postfix}.
\item
  Karena menggunakan \texttt{postfix} maka yang dilakukan pada operator
  \texttt{++} adalah mengkopi objek yang lama, menambahkan data,
  kemudian mengembalikan objek tersebut by value.
\end{itemize}

\section{Conversion Operator}\label{conversion-operator}

Bagaimana jika anda menginginkan statement
\texttt{int\ bil\ =\ Kalender(23,10,2010)} memiliki arti? Untuk itu anda
perlu mengkonversi dari objek Kalender menjadi tipe data integer, dengan
demikian anda akan dengan mudah mengirimkan data Kalender ke module lain
yang hanya menerima parameter bertipe integer.

Anda dapat melakukan konversi diatas dengan menggunakan conversion
operator yang mempunyai syntax sebagai berikut:

\begin{lstlisting}[language=c++, numbers=none]
operator conversion_type();
\end{lstlisting}

Jadi jika anda menghendaki mengkonversi tipe Kalender menjadi int anda
dapat menggunakan operator berikut.

\begin{lstlisting}[language=c++, numbers=none]
operator int()
{
// implementation
return intValue;
}
\end{lstlisting}

Contoh \ref{contoh8-3} dibawah ini akan menunjukan bagaimana penggunaan conversion
operator untuk mengkonversi tipe Kalender menjadi int.

\subsubsection*{Contoh Conversion Operator untuk konversi class Kalender ke integer.}

Buka Qt Creator dan buat project Qt Console Application baru dengan nama
contoh \ref{contoh8-3}, kemudian tulis kode berikut.

\lstinputlisting[language=c++, caption=Conversion Operator untuk konversi class Kalender ke integer, label=contoh8-3]{../code/contoh8-3.cpp}



\textbf{Hasilnya}
\begin{lcverbatim}
Inisialisasi data :
23 / 10 / 2010

Integer yang sesuai dengan data 2011023
\end{lcverbatim}


\subsubsection*{Keterangan}

\begin{itemize}

\item
  Pada operator \texttt{int()}, \texttt{variabel} \texttt{\_tahun},
  \texttt{\_bulan}, dan \texttt{\_hari} dikalikan dengan bilangan
  tertentu sehingga menghasilkan kembalian berupa \texttt{int}.
\item
  Statement int \texttt{nData\ =\ objKal} akan menjalankan operator
  \texttt{int} dan mengembalikan nilai integer dari objek Kalender.
\item
  Dengan menggunakan operator int akan lebih mudah membandingkan dua
  objek Kalender, karena objek tersebut dapat mengembalikan satu nilai
  integer.
\end{itemize}

\subsection{Binary Operator}\label{binary-operator}

Operator yang mengoperasikan dua operand disebut dengan binary operator,
cara penulisan binary operator sama dengan penulisan oprator yang
sebelumnya.

\begin{lstlisting}[language=c++, numbers=none]
return_type operator_type (parameter);
\end{lstlisting}

Ada beberapa macam binary operator yang dapat digunakan pada C++,
diantaranya :

\begin{longtable}[]{@{}ll@{}}
\toprule
Operator & Name\tabularnewline
\midrule
\endhead
+ & Addition\tabularnewline
+= & Addition/Assigment\tabularnewline
- & Substraction\tabularnewline
-= & Substraction/Assigment\tabularnewline
\textless{} & Less Than\tabularnewline
\textgreater{} & Greater Than\tabularnewline
\textless{}= & Less thar or equal to\tabularnewline
\textgreater{}= & Greater than or equal to\tabularnewline
== & Equal to\tabularnewline
!= & Inequality\tabularnewline
\bottomrule
\end{longtable}

Contoh \ref{contoh8-4} program dibawah ini menggunakan operator Addition (+) untuk
menambahkan hari pada objek kalender, anda dapat menambahkan beberapa
hari kedepan, misal 5 atau 10 hari dari tanggal sekarang.

\subsubsection*{Contoh  Menggunakan Binary Addition Operator.}

\begin{enumerate}

\item
  Buka Qt Creator dan buat project Qt Console Application baru dengan
  nama contoh \ref{contoh8-4}, kemudian tulis kode berikut.

\lstinputlisting[language=c++, caption=Menggunakan Binary Addition Operator, label=contoh8-4]{../code/contoh8-4.cpp}

\item
  Kemudian jalankan kode diatas dengan menekan tombol Ctrl+R, outputnya
  adalah sebagai berikut.
\end{enumerate}

\begin{lcverbatim}
Inisialisasi Data

23 / 10 / 2011 Menambahkan 25 hari kedepan
Hasil setelah ditambahkan 25 hari 18 / 11 / 2011 
13 / 11 / 2011
\end{lcverbatim}

\subsubsection*{Keterangan}

\begin{itemize}

\item
  Dengan menggunakan operator + anda dapat menambahkan hari pada objek
  Kalender, jumlah hari yang ditambahkan tergantung dari nilai yang
  diinputkan pada parameter.
\item
  Anda dapat menambah hari dengan menggunakan operator + pada objek
  Kalender, misal: \texttt{objKal\ =\ objKal\ +\ 25} atau dengan membuat
  objek baru untuk menampung nilai hasil penambahan Kalender
  \texttt{objKalBaru(objKal+20)}
\end{itemize}

\section{Addition-Assignment
Operator}\label{addition-assignment-operator}

Dengan menggunakan Addition-Assignment operator anda dapat menuliskan
sintaks a += b, yang sama artinya dengan a = a + b. Pada contoh program
dibawah ini operator \emph{Addition-Assignment} akan digunakan untuk
menambahkan hari pada objek Kalender.

\subsubsection*{Contoh  Menggunakan Addition Assigment Operator dan Substraction Assigment Operator.}

\begin{enumerate}

\item
  Buka Qt Creator dan buat project Qt Console Application baru dengan
  nama contoh \ref{contoh8-5}, kemudian tulis kode berikut.

\lstinputlisting[language=c++, caption= Menggunakan Addition Assigment Operator dan Substraction
Assigment Operator, label=contoh8-5]{../code/contoh8-5.cpp}


\item
  Kemudian jalankan kode diatas dengan menekan tombol Ctrl+R, outputnya
  adalah sebagai berikut.
\end{enumerate}

\begin{lcverbatim}
Inisialisasi Data
23/10/2011
Menambahkan 25 hari kedepan
Hasil setelah ditambahkan 25 hari
18/11/2011
\end{lcverbatim}

\subsubsection*{Keterangan}

Pada kode diatas anda dapat menggunakan operator Addition-Assignment
pada objek Kalender, misal anda ingin menambahkan 25 hari pada objek
Kalender, anda dapat menuliskan kode \texttt{objKal\ +=\ 25;}

\section{Comparison Operator}\label{comparison-operator}

Pada kasus tertentu dimana anda ingin membandingkan dua objek bertipe
Kalender ada dapat menggunakan comparison operator.

\begin{lstlisting}[language=c++, numbers=none]
if (objKal1 == objKal2)
{
// Do something
}
else
{
// Do something else
}
\end{lstlisting}

Anda dapat menggunakan equality operator (==) atau inequality operator
(!=). Anda juga dapat membuat lebih dari satu equality atau inequality
operator yang mempunyai return value atau parameter yang berbeda, ini
disebut dengan overloading operator.

\subsubsection*{Contoh  Overloading Comparison Operator (Equality dan Inequality).}

\begin{enumerate}

\item
  Buka Qt Creator dan buat project Qt Console Application baru dengan
  nama contoh \ref{contoh8-6}, kemudian tulis kode berikut.

\lstinputlisting[language=c++, caption=Overloading Comparison Operator (Equality dan Inequality, label=contoh8-6]{../code/contoh8-6.cpp}


\item
  Kemudian jalankan kode diatas dengan menekan tombol Ctrl+R, outputnya
  adalah sebagai berikut.
\end{enumerate}

\begin{lcverbatim}
Inisialisasi Kalender 1
23 / 10 / 2010

Inisialisasi Kalender 2
23 / 10 / 2011

kalender1 dan kalender2 tidak sama !
Inisialisasi Kalender 3
23 / 10 / 2010

kalender1 dan kalender3 sama !
nilai integer yang ekuivalen dengan objKal3 adalah 20101023
Nilai integer dari objKal3 dan intKal3 sama
Nilai integer dari objKal2 dan intKal3 tidak sama
\end{lcverbatim}

\subsubsection*{Keterangan}

\begin{itemize}

\item
  Pada kode diatas terdapat 4 comparison operator yang berbeda, walaupun
  masing-masing ada 2 \emph{inequality} dan \emph{equality} operator
  namun parameternya berbeda ini disebut sebagai \emph{overloading
  operator}.
\item
  Pada operator equality (==) yang pertama membandingkan dua objek
  Kalender dengan cara membandingkan member variabel hari, bulan, tahun
  pada masing-masing objek yang dibandingkan, sedangkan operator == yang
  kedua membandingkan objek Kalender yang terlebih dahulu sudah
  dikonversi menjadi int.
\item
  Anda dapat melihat bahwa kedua operator comparison diatas sama-sama
  dapat membandingkan isi dari 2 objek Kalender, baik dengan cara
  membandingkan member variabel maupun membandingkan nilai int (hasil
  konversi dari objek Kalender).
\end{itemize}

\section{Overloading Operator \textless{}, \textgreater{}, \textless{}=, \textgreater{}=}\label{overloading-operator}

Seperti pada contoh \ref{contoh8-6} sebelumnya anda juga dapat menggunakan operator
\textless{}, \textgreater{}, \textless{}=, \textgreater{}= untuk
membandingkan objek Kalender. Agar mudah untuk dibandingkan maka objek
Kalender dikonversi terlebih dahulu menjadi tipe int.

\subsubsection*{Contoh  Menggunakan Operator \textless{}. \textgreater{}, \textless{}=, \textgreater{}=.}
\begin{enumerate}

\item
  Buka Qt Creator dan buat project Qt Console Application baru dengan
  nama contoh \ref{contoh8-7}, kemudian tulis kode berikut.

\lstinputlisting[language=c++, caption=Menggunakan Operator Overloading, label=contoh8-7]{../code/contoh8-7.cpp}


\item
  Kemudian jalankan kode diatas dengan menekan tombol Ctrl+R, outputnya
  adalah sebagai berikut.
\end{enumerate}

\begin{lcverbatim}
objKal1 berisi :
23 / 10 / 2010

objKal2 berisi :
16 / 10 / 1980

objKal3 berisi :
23 / 10 / 2010

objKal1 < objKal2 = false
objKal1 > objKal2 = true
objKal1 <= objKal3 = true
objKal1 >= objKal3 = true
\end{lcverbatim}

\subsubsection*{Keterangan}

\begin{itemize}

\item
  Pada program diatas penggunaan operator \textless{}, \textgreater{},
  \textless{}=, \textgreater{}= digunakan untuk membandingkan dua objek
  Kalender yang berbeda.
\item
  Untuk mempermudah membandingkan dua objek Kalender maka objek Kalender
  tersebut dikonversi terlebih dahulu menjadi \texttt{int}.
\end{itemize}

\section{Subscript Operator}\label{subscript-operator}

Subscript operator dapat digunakan jika anda ingin mengakses class
seperti ketika anda mengakses array, anda dapat menambahkan operator
\texttt{{[}{]}} pada objek yang anda buat untuk mengakses nilai dengan
index tertentu dari objek. Contoh dibawah ini akan menjelaskan
penggunaan subscript operator untuk membuat array yang dinamis.

\subsubsection*{Contoh  Subscript Operator untuk Dynamic Array.}

\begin{enumerate}

\item
  Buka Qt Creator dan buat project Qt Console Application baru dengan
  nama contoh \ref{contoh8-8}, kemudian tulis kode berikut.

\lstinputlisting[language=c++, caption=Subscript Operator untuk Dynamic Array, label=contoh8-8]{../code/contoh8-8.cpp}


\item
  Kemudian jalankan kode diatas dengan menekan tombol Ctrl+R, outputnya
  adalah sebagai berikut.
\end{enumerate}

\begin{lcverbatim}
The content array are : {23 16 9 20 55 }
\end{lcverbatim}

\subsubsection*{Keterangan}

Dengan menggunakan subscript operator anda dapat mengakses member
variable yang bertipe array pada class dengan menggunakan array-like
syntax (nama objek diikuti dengan tanda {[}{]}), misal:
\texttt{namaObjek{[}index{]}}.

\section{Function operator()}\label{function-operator}

Function operator digunakan jika anda ingin membuat objek bekerja
seperti function. Untuk lebih jelasnya penggunaan function
\texttt{operator()} anda dapat mencoba program dibawah ini.

\subsubsection*{Contoh  Menggunakan operator() untuk membuat function object.}

\begin{enumerate}

\item
  Buka Qt Creator dan buat project Qt Console Application baru dengan
  nama contoh \ref{contoh8-9}, kemudian tulis kode berikut.

\lstinputlisting[language=c++, caption=Menggunakan operator() untuk membuat function object, label=contoh8-9]{../code/contoh8-9.cpp}


\item
  Kemudian jalankan kode diatas dengan menekan tombol Ctrl+R, outputnya
  adalah sebagai berikut.
\end{enumerate}

\begin{lcverbatim}
Hello Function Operator !
\end{lcverbatim}

\subsubsection*{Analisis}

\begin{itemize}

\item
  Dengan membuat function operator maka anda dapat menggunakan objek
  seperti ketika anda menggunakan function.
\item
  Pada program diatas objek objTampil dapat dipanggil seperti function.
\end{itemize}
