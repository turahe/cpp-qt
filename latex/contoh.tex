\chapter{Contoh Program Lengkap}

Bab ini berisi kumpulan contoh program C++ yang dapat digunakan sebagai referensi dan latihan. Program-program ini mencakup berbagai konsep yang telah dipelajari dalam buku ini.

\section{Program Dasar}

\subsection{Contoh A: Mencari Nilai Minimum}

Program untuk mencari nilai minimum dari 10 bilangan bulat yang diinputkan oleh user.

\begin{lstlisting}[language=c++, caption=Program Mencari Nilai Minimum]
\lstinputlisting[language=c++]{../code/contoh-a.cpp}
\end{lstlisting}

\subsection{Contoh B: Menentukan Bilangan Genap/Ganjil}

Program untuk menginputkan sebuah bilangan dan mengenalinya apakah bilangan tersebut genap atau ganjil.

\begin{lstlisting}[language=c++, caption=Program Menentukan Bilangan Genap/Ganjil]
\lstinputlisting[language=c++]{../code/contoh-b.cpp}
\end{lstlisting}

\section{Program Tipe Data dan Variabel}

\subsection{Contoh 2-1: Deklarasi dan Inisialisasi Variabel}

Program sederhana untuk mendemonstrasikan deklarasi dan inisialisasi variabel.

\begin{lstlisting}[language=c++, caption=Deklarasi dan Inisialisasi Variabel]
\lstinputlisting[language=c++]{../code/contoh2-1.cpp}
\end{lstlisting}

\subsection{Contoh 2-2: Operasi Aritmatika}

Program untuk melakukan operasi aritmatika dasar.

\begin{lstlisting}[language=c++, caption=Operasi Aritmatika]
\lstinputlisting[language=c++]{../code/contoh2-2.cpp}
\end{lstlisting}

\subsection{Contoh 2-3: Konversi Tipe Data}

Program untuk mendemonstrasikan konversi tipe data.

\begin{lstlisting}[language=c++, caption=Konversi Tipe Data]
\lstinputlisting[language=c++]{../code/contoh2-3.cpp}
\end{lstlisting}

\subsection{Contoh 2-4: Penggunaan Konstanta}

Program untuk mendemonstrasikan penggunaan konstanta.

\begin{lstlisting}[language=c++, caption=Penggunaan Konstanta]
\lstinputlisting[language=c++]{../code/contoh2-4.cpp}
\end{lstlisting}

\subsection{Contoh 2-5: Break dan Continue}

Program untuk mendemonstrasikan penggunaan break dan continue.

\begin{lstlisting}[language=c++, caption=Break dan Continue]
\lstinputlisting[language=c++]{../code/contoh2-5.cpp}
\end{lstlisting}

\section{Program Control Statement}

\subsection{Contoh 3-1: If Statement}

Program untuk mendemonstrasikan penggunaan if statement.

\begin{lstlisting}[language=c++, caption=If Statement]
\lstinputlisting[language=c++]{../code/contoh3-1.cpp}
\end{lstlisting}

\subsection{Contoh 3-2: If-Else Statement}

Program untuk mendemonstrasikan penggunaan if-else statement.

\begin{lstlisting}[language=c++, caption=If-Else Statement]
\lstinputlisting[language=c++]{../code/contoh3-2.cpp}
\end{lstlisting}

\subsection{Contoh 3-3: Switch Statement}

Program untuk mendemonstrasikan penggunaan switch statement.

\begin{lstlisting}[language=c++, caption=Switch Statement]
\lstinputlisting[language=c++]{../code/contoh3-3.cpp}
\end{lstlisting}

\subsection{Contoh 3-4: For Loop}

Program untuk mendemonstrasikan penggunaan for loop.

\begin{lstlisting}[language=c++, caption=For Loop]
\lstinputlisting[language=c++]{../code/contoh3-4.cpp}
\end{lstlisting}

\subsection{Contoh 3-5: While Loop}

Program untuk mendemonstrasikan penggunaan while loop.

\begin{lstlisting}[language=c++, caption=While Loop]
\lstinputlisting[language=c++]{../code/contoh3-5.cpp}
\end{lstlisting}

\section{Program Array dan String}

\subsection{Contoh 4-1: Array Satu Dimensi}

Program untuk mendemonstrasikan penggunaan array satu dimensi.

\begin{lstlisting}[language=c++, caption=Array Satu Dimensi]
\lstinputlisting[language=c++]{../code/contoh4-1.cpp}
\end{lstlisting}

\subsection{Contoh 4-2: Array Dua Dimensi}

Program untuk mendemonstrasikan penggunaan array dua dimensi.

\begin{lstlisting}[language=c++, caption=Array Dua Dimensi]
\lstinputlisting[language=c++]{../code/contoh4-2.cpp}
\end{lstlisting}

\subsection{Contoh 4-3: Manipulasi String}

Program untuk mendemonstrasikan manipulasi string.

\begin{lstlisting}[language=c++, caption=Manipulasi String]
\lstinputlisting[language=c++]{../code/contoh4-3.cpp}
\end{lstlisting}

\subsection{Contoh 4-4: String Functions}

Program untuk mendemonstrasikan fungsi-fungsi string.

\begin{lstlisting}[language=c++, caption=String Functions]
\lstinputlisting[language=c++]{../code/contoh4-4.cpp}
\end{lstlisting}

\subsection{Contoh 4-5: String Comparison}

Program untuk mendemonstrasikan perbandingan string.

\begin{lstlisting}[language=c++, caption=String Comparison]
\lstinputlisting[language=c++]{../code/contoh4-5.cpp}
\end{lstlisting}

\section{Program Fungsi}

\subsection{Contoh 7-1: Fungsi Sederhana}

Program untuk mendemonstrasikan pembuatan fungsi sederhana.

\begin{lstlisting}[language=c++, caption=Fungsi Sederhana]
\lstinputlisting[language=c++]{../code/contoh7-1.cpp}
\end{lstlisting}

\subsection{Contoh 7-3: Fungsi dengan Parameter}

Program untuk mendemonstrasikan fungsi dengan parameter.

\begin{lstlisting}[language=c++, caption=Fungsi dengan Parameter]
\lstinputlisting[language=c++]{../code/contoh7-3.cpp}
\end{lstlisting}

\subsection{Contoh 7-4: Fungsi dengan Return Value}

Program untuk mendemonstrasikan fungsi dengan return value.

\begin{lstlisting}[language=c++, caption=Fungsi dengan Return Value]
\lstinputlisting[language=c++]{../code/contoh7-4.cpp}
\end{lstlisting}

\subsection{Contoh 7-5: Function Overloading}

Program untuk mendemonstrasikan function overloading.

\begin{lstlisting}[language=c++, caption=Function Overloading]
\lstinputlisting[language=c++]{../code/contoh7-5.cpp}
\end{lstlisting}

\subsection{Contoh 7-6: Recursive Function}

Program untuk mendemonstrasikan recursive function.

\begin{lstlisting}[language=c++, caption=Recursive Function]
\lstinputlisting[language=c++]{../code/contoh7-6.cpp}
\end{lstlisting}

\subsection{Contoh 7-7: Default Parameters}

Program untuk mendemonstrasikan default parameters.

\begin{lstlisting}[language=c++, caption=Default Parameters]
\lstinputlisting[language=c++]{../code/contoh7-7.cpp}
\end{lstlisting}

\subsection{Contoh 7-9: Inline Function}

Program untuk mendemonstrasikan inline function.

\begin{lstlisting}[language=c++, caption=Inline Function]
\lstinputlisting[language=c++]{../code/contoh7-9.cpp}
\end{lstlisting}

\subsection{Contoh 7-10: Template Function}

Program untuk mendemonstrasikan template function.

\begin{lstlisting}[language=c++, caption=Template Function]
\lstinputlisting[language=c++]{../code/contoh7-10.cpp}
\end{lstlisting}

\subsection{Contoh 7-11: Lambda Function}

Program untuk mendemonstrasikan lambda function.

\begin{lstlisting}[language=c++, caption=Lambda Function]
\lstinputlisting[language=c++]{../code/contoh7-11.cpp}
\end{lstlisting}

\subsection{Contoh 7-12: Function Pointer}

Program untuk mendemonstrasikan function pointer.

\begin{lstlisting}[language=c++, caption=Function Pointer]
\lstinputlisting[language=c++]{../code/contoh7-12.cpp}
\end{lstlisting}

\subsection{Contoh 7-13: Callback Function}

Program untuk mendemonstrasikan callback function.

\begin{lstlisting}[language=c++, caption=Callback Function]
\lstinputlisting[language=c++]{../code/contoh7-13.cpp}
\end{lstlisting}

\section{Program Pointer dan References}

\subsection{Contoh 8-1: Basic Pointer}

Program untuk mendemonstrasikan penggunaan pointer dasar.

\begin{lstlisting}[language=c++, caption=Basic Pointer]
\lstinputlisting[language=c++]{../code/contoh8-1.cpp}
\end{lstlisting}

\subsection{Contoh 8-2: Pointer Arithmetic}

Program untuk mendemonstrasikan pointer arithmetic.

\begin{lstlisting}[language=c++, caption=Pointer Arithmetic]
\lstinputlisting[language=c++]{../code/contoh8-2.cpp}
\end{lstlisting}

\subsection{Contoh 8-3: Pointer to Array}

Program untuk mendemonstrasikan pointer to array.

\begin{lstlisting}[language=c++, caption=Pointer to Array]
\lstinputlisting[language=c++]{../code/contoh8-3.cpp}
\end{lstlisting}

\subsection{Contoh 8-4: Pointer to Function}

Program untuk mendemonstrasikan pointer to function.

\begin{lstlisting}[language=c++, caption=Pointer to Function]
\lstinputlisting[language=c++]{../code/contoh8-4.cpp}
\end{lstlisting}

\subsection{Contoh 8-5: Void Pointer}

Program untuk mendemonstrasikan void pointer.

\begin{lstlisting}[language=c++, caption=Void Pointer]
\lstinputlisting[language=c++]{../code/contoh8-5.cpp}
\end{lstlisting}

\subsection{Contoh 8-6: Reference}

Program untuk mendemonstrasikan penggunaan reference.

\begin{lstlisting}[language=c++, caption=Reference]
\lstinputlisting[language=c++]{../code/contoh8-6.cpp}
\end{lstlisting}

\subsection{Contoh 8-7: Reference as Parameter}

Program untuk mendemonstrasikan reference sebagai parameter.

\begin{lstlisting}[language=c++, caption=Reference as Parameter]
\lstinputlisting[language=c++]{../code/contoh8-7.cpp}
\end{lstlisting}

\subsection{Contoh 8-8: Const Reference}

Program untuk mendemonstrasikan const reference.

\begin{lstlisting}[language=c++, caption=Const Reference]
\lstinputlisting[language=c++]{../code/contoh8-8.cpp}
\end{lstlisting}

\subsection{Contoh 8-9: Reference Return}

Program untuk mendemonstrasikan reference return.

\begin{lstlisting}[language=c++, caption=Reference Return]
\lstinputlisting[language=c++]{../code/contoh8-9.cpp}
\end{lstlisting}

\section{Program Class dan Object}

\subsection{Contoh 9-1: Basic Class}

Program untuk mendemonstrasikan pembuatan class dasar.

\begin{lstlisting}[language=c++, caption=Basic Class]
\lstinputlisting[language=c++]{../code/contoh9-1.cpp}
\end{lstlisting}

\subsection{Contoh 9-2: Constructor dan Destructor}

Program untuk mendemonstrasikan constructor dan destructor.

\begin{lstlisting}[language=c++, caption=Constructor dan Destructor]
\lstinputlisting[language=c++]{../code/contoh9-2.cpp}
\end{lstlisting}

\subsection{Contoh 9-3: Access Specifiers}

Program untuk mendemonstrasikan access specifiers.

\begin{lstlisting}[language=c++, caption=Access Specifiers]
\lstinputlisting[language=c++]{../code/contoh9-3.cpp}
\end{lstlisting}

\subsection{Contoh 9-4: Static Members}

Program untuk mendemonstrasikan static members.

\begin{lstlisting}[language=c++, caption=Static Members]
\lstinputlisting[language=c++]{../code/contoh9-4.cpp}
\end{lstlisting}

\subsection{Contoh 9-5: Friend Function}

Program untuk mendemonstrasikan friend function.

\begin{lstlisting}[language=c++, caption=Friend Function]
\lstinputlisting[language=c++]{../code/contoh9-5.cpp}
\end{lstlisting}

\subsection{Contoh 9-6: Operator Overloading}

Program untuk mendemonstrasikan operator overloading.

\begin{lstlisting}[language=c++, caption=Operator Overloading]
\lstinputlisting[language=c++]{../code/contoh9-6.cpp}
\end{lstlisting}

\section{Program Inheritance}

\subsection{Contoh 12-1: Single Inheritance}

Program untuk mendemonstrasikan single inheritance.

\begin{lstlisting}[language=c++, caption=Single Inheritance]
\lstinputlisting[language=c++]{../code/contoh12-1.cpp}
\end{lstlisting}

\subsection{Contoh 12-2: Multiple Inheritance}

Program untuk mendemonstrasikan multiple inheritance.

\begin{lstlisting}[language=c++, caption=Multiple Inheritance]
\lstinputlisting[language=c++]{../code/contoh12-2.cpp}
\end{lstlisting}

\subsection{Contoh 12-3: Multilevel Inheritance}

Program untuk mendemonstrasikan multilevel inheritance.

\begin{lstlisting}[language=c++, caption=Multilevel Inheritance]
\lstinputlisting[language=c++]{../code/contoh12-3.cpp}
\end{lstlisting}

\section{Program Polymorphism}

\subsection{Contoh Polymorphism 1: Virtual Function}

Program untuk mendemonstrasikan virtual function.

\begin{lstlisting}[language=c++, caption=Virtual Function]
\lstinputlisting[language=c++]{../code/09-Polymorphism-code-7.cpp}
\end{lstlisting}

\subsection{Contoh Polymorphism 2: Pure Virtual Function}

Program untuk mendemonstrasikan pure virtual function.

\begin{lstlisting}[language=c++, caption=Pure Virtual Function]
\lstinputlisting[language=c++]{../code/09-Polymorphism-code-8.cpp}
\end{lstlisting}

\subsection{Contoh Polymorphism 3: Abstract Class}

Program untuk mendemonstrasikan abstract class.

\begin{lstlisting}[language=c++, caption=Abstract Class]
\lstinputlisting[language=c++]{../code/09-Polymorphism-code-9.cpp}
\end{lstlisting}

\section{Program Casting}

\subsection{Contoh Casting 1: Static Cast}

Program untuk mendemonstrasikan static cast.

\begin{lstlisting}[language=c++, caption=Static Cast]
\lstinputlisting[language=c++]{../code/10-casting-database-code-1.cpp}
\end{lstlisting}

\subsection{Contoh Casting 2: Dynamic Cast}

Program untuk mendemonstrasikan dynamic cast.

\begin{lstlisting}[language=c++, caption=Dynamic Cast]
\lstinputlisting[language=c++]{../code/10-casting-database-contoh-dynamic-casting.cpp}
\end{lstlisting}

\subsection{Contoh Casting 3: Explicit Casting}

Program untuk mendemonstrasikan explicit casting.

\begin{lstlisting}[language=c++, caption=Explicit Casting]
\lstinputlisting[language=c++]{../code/10-casting-database-explicit-casting-pada-tipe-data-numerik.cpp}
\end{lstlisting}

\section{Program Database}

\subsection{Contoh Database 1: Koneksi SQLite}

Program untuk mendemonstrasikan koneksi ke SQLite.

\begin{lstlisting}[language=c++, caption=Koneksi SQLite]
\lstinputlisting[language=c++]{../code/10-casting-database-percobaan-koneksi-sqlite-dengan-qtconsole.cpp}
\end{lstlisting}

\subsection{Contoh Database 2: Koneksi MySQL}

Program untuk mendemonstrasikan koneksi ke MySQL.

\begin{lstlisting}[language=c++, caption=Koneksi MySQL]
\lstinputlisting[language=c++]{../code/10-casting-database-percobaan-koneksi-mysql-dengan-qtconsole.cpp}
\end{lstlisting}

\subsection{Contoh Database 3: Menambah Data SQLite}

Program untuk mendemonstrasikan menambah data ke SQLite.

\begin{lstlisting}[language=c++, caption=Menambah Data SQLite]
\lstinputlisting[language=c++]{../code/10-casting-database-menambahkan-data-pada-sqlite.cpp}
\end{lstlisting}

\subsection{Contoh Database 4: Membaca Data SQLite}

Program untuk mendemonstrasikan membaca data dari SQLite.

\begin{lstlisting}[language=c++, caption=Membaca Data SQLite]
\lstinputlisting[language=c++]{../code/10-casting-database-membaca-data-pada-sqlite.cpp}
\end{lstlisting}

\subsection{Contoh Database 5: Mengedit Data SQLite}

Program untuk mendemonstrasikan mengedit data di SQLite.

\begin{lstlisting}[language=c++, caption=Mengedit Data SQLite]
\lstinputlisting[language=c++]{../code/10-casting-database-mengedit-data-pada-sqlite.cpp}
\end{lstlisting}

\subsection{Contoh Database 6: Menghapus Data SQLite}

Program untuk mendemonstrasikan menghapus data dari SQLite.

\begin{lstlisting}[language=c++, caption=Menghapus Data SQLite]
\lstinputlisting[language=c++]{../code/10-casting-database-menghapus-data-pada-sqlite.cpp}
\end{lstlisting}

\section{Program Qt Library}

\subsection{Contoh Qt Library 1: QObject dan Memory Management}

Program untuk mendemonstrasikan QObject dan memory management.

\begin{lstlisting}[language=c++, caption=QObject dan Memory Management]
\lstinputlisting[language=c++]{../code/11-library-contoh11-1.cpp}
\end{lstlisting}

\subsection{Contoh Qt Library 2: QString Operations}

Program untuk mendemonstrasikan operasi QString.

\begin{lstlisting}[language=c++, caption=QString Operations]
\lstinputlisting[language=c++]{../code/11-library-contoh11-2.cpp}
\end{lstlisting}

\subsection{Contoh Qt Library 3: QList Container}

Program untuk mendemonstrasikan penggunaan QList.

\begin{lstlisting}[language=c++, caption=QList Container]
\lstinputlisting[language=c++]{../code/11-library-contoh11-3.cpp}
\end{lstlisting}

\subsection{Contoh Qt Library 4: QMap Container}

Program untuk mendemonstrasikan penggunaan QMap.

\begin{lstlisting}[language=c++, caption=QMap Container]
\lstinputlisting[language=c++]{../code/11-library-contoh11-4.cpp}
\end{lstlisting}

\subsection{Contoh Qt Library 5: QStack dan QQueue}

Program untuk mendemonstrasikan QStack dan QQueue.

\begin{lstlisting}[language=c++, caption=QStack dan QQueue]
\lstinputlisting[language=c++]{../code/11-library-contoh11-5.cpp}
\end{lstlisting}

\subsection{Contoh Qt Library 6: Iterator}

Program untuk mendemonstrasikan penggunaan iterator.

\begin{lstlisting}[language=c++, caption=Iterator]
\lstinputlisting[language=c++]{../code/11-library-contoh11-6.cpp}
\end{lstlisting}

\subsection{Contoh Qt Library 7: Lambda dengan Qt}

Program untuk mendemonstrasikan lambda dengan Qt.

\begin{lstlisting}[language=c++, caption=Lambda dengan Qt]
\lstinputlisting[language=c++]{../code/11-library-contoh11-7.cpp}
\end{lstlisting}

\subsection{Contoh Qt Library 8: Signal dan Slot}

Program untuk mendemonstrasikan signal dan slot.

\begin{lstlisting}[language=c++, caption=Signal dan Slot]
\lstinputlisting[language=c++]{../code/11-library-contoh11-8.cpp}
\end{lstlisting}

\subsection{Contoh Qt Library 9: Custom Signal}

Program untuk mendemonstrasikan custom signal.

\begin{lstlisting}[language=c++, caption=Custom Signal]
\lstinputlisting[language=c++]{../code/11-library-contoh11-9.cpp}
\end{lstlisting}

\subsection{Contoh Qt Library 10: QTimer}

Program untuk mendemonstrasikan penggunaan QTimer.

\begin{lstlisting}[language=c++, caption=QTimer]
\lstinputlisting[language=c++]{../code/11-library-contoh11-10.cpp}
\end{lstlisting}

\subsection{Contoh Qt Library 11: QThread}

Program untuk mendemonstrasikan penggunaan QThread.

\begin{lstlisting}[language=c++, caption=QThread]
\lstinputlisting[language=c++]{../code/11-library-contoh11-11.cpp}
\end{lstlisting}

\subsection{Contoh Qt Library 12: QFile dan QDir}

Program untuk mendemonstrasikan QFile dan QDir.

\begin{lstlisting}[language=c++, caption=QFile dan QDir]
\lstinputlisting[language=c++]{../code/11-library-contoh11-12.cpp}
\end{lstlisting}

\subsection{Contoh Qt Library 13: QSettings}

Program untuk mendemonstrasikan penggunaan QSettings.

\begin{lstlisting}[language=c++, caption=QSettings]
\lstinputlisting[language=c++]{../code/11-library-contoh11-13.cpp}
\end{lstlisting}

\section{Program File, Stream, dan XML}

\subsection{Contoh File 1: Membaca File Teks}

Program untuk mendemonstrasikan membaca file teks.

\begin{lstlisting}[language=c++, caption=Membaca File Teks]
\lstinputlisting[language=c++]{../code/12-file-stream-xml-contoh12-4.cpp}
\end{lstlisting}

\subsection{Contoh File 2: Menulis File Teks}

Program untuk mendemonstrasikan menulis file teks.

\begin{lstlisting}[language=c++, caption=Menulis File Teks]
\lstinputlisting[language=c++]{../code/12-file-stream-xml-contoh12-5.cpp}
\end{lstlisting}

\subsection{Contoh File 3: File Biner}

Program untuk mendemonstrasikan operasi file biner.

\begin{lstlisting}[language=c++, caption=File Biner]
\lstinputlisting[language=c++]{../code/12-file-stream-xml-contoh12-6.cpp}
\end{lstlisting}

\subsection{Contoh File 4: Directory Operations}

Program untuk mendemonstrasikan operasi direktori.

\begin{lstlisting}[language=c++, caption=Directory Operations]
\lstinputlisting[language=c++]{../code/12-file-stream-xml-contoh12-7.cpp}
\end{lstlisting}

\subsection{Contoh File 5: File Info}

Program untuk mendemonstrasikan informasi file.

\begin{lstlisting}[language=c++, caption=File Info]
\lstinputlisting[language=c++]{../code/12-file-stream-xml-contoh12-8.cpp}
\end{lstlisting}

\subsection{Contoh File 6: File Permissions}

Program untuk mendemonstrasikan permission file.

\begin{lstlisting}[language=c++, caption=File Permissions]
\lstinputlisting[language=c++]{../code/12-file-stream-xml-contoh12-9.cpp}
\end{lstlisting}

\section{Program XML}

\subsection{Contoh XML 1: Membuat XML dengan DOM}

Program untuk mendemonstrasikan membuat XML dengan DOM.

\begin{lstlisting}[language=c++, caption=Membuat XML dengan DOM]
\lstinputlisting[language=c++]{../code/12-file-stream-xml-code-4.cpp}
\end{lstlisting}

\subsection{Contoh XML 2: Membaca XML dengan DOM}

Program untuk mendemonstrasikan membaca XML dengan DOM.

\begin{lstlisting}[language=c++, caption=Membaca XML dengan DOM]
\lstinputlisting[language=c++]{../code/12-file-stream-xml-code-12.cpp}
\end{lstlisting}

\subsection{Contoh XML 3: QXMLStreamReader}

Program untuk mendemonstrasikan QXMLStreamReader.

\begin{lstlisting}[language=c++, caption=QXMLStreamReader]
\lstinputlisting[language=c++]{../code/12-file-stream-xml-code-6.xml}
\end{lstlisting}

\subsection{Contoh XML 4: QXMLStreamWriter}

Program untuk mendemonstrasikan QXMLStreamWriter.

\begin{lstlisting}[language=c++, caption=QXMLStreamWriter]
\lstinputlisting[language=c++]{../code/12-file-stream-xml-code-8.xml}
\end{lstlisting}

\section{Program Qt WebKit}

\subsection{Contoh WebKit 1: Browser Sederhana}

Program untuk mendemonstrasikan browser sederhana dengan Qt WebKit.

\begin{lstlisting}[language=c++, caption=Browser Sederhana]
\lstinputlisting[language=c++]{../code/13-qt-webkit-code-1.c++}
\end{lstlisting}

\subsection{Contoh WebKit 2: Browser dengan Kontrol}

Program untuk mendemonstrasikan browser dengan kontrol navigasi.

\begin{lstlisting}[language=c++, caption=Browser dengan Kontrol]
\lstinputlisting[language=c++]{../code/13-qt-webkit-code-2.c++}
\end{lstlisting}

\subsection{Contoh WebKit 3: URL Bar}

Program untuk mendemonstrasikan URL bar dalam browser.

\begin{lstlisting}[language=c++, caption=URL Bar]
\lstinputlisting[language=c++]{../code/13-qt-webkit-code-3.c++}
\end{lstlisting}

\subsection{Contoh WebKit 4: Progress Bar}

Program untuk mendemonstrasikan progress bar dalam browser.

\begin{lstlisting}[language=c++, caption=Progress Bar]
\lstinputlisting[language=c++]{../code/13-qt-webkit-code-5.c++}
\end{lstlisting}

\subsection{Contoh WebKit 5: JavaScript Integration}

Program untuk mendemonstrasikan integrasi JavaScript.

\begin{lstlisting}[language=c++, caption=JavaScript Integration]
\lstinputlisting[language=c++]{../code/13-qt-webkit-code-6.c++}
\end{lstlisting}

\subsection{Contoh WebKit 6: Custom User Agent}

Program untuk mendemonstrasikan custom user agent.

\begin{lstlisting}[language=c++, caption=Custom User Agent]
\lstinputlisting[language=c++]{../code/13-qt-webkit-code-7.c++}
\end{lstlisting}

\subsection{Contoh WebKit 7: Web Inspector}

Program untuk mendemonstrasikan web inspector.

\begin{lstlisting}[language=c++, caption=Web Inspector]
\lstinputlisting[language=c++]{../code/13-qt-webkit-code-8.c++}
\end{lstlisting}

\subsection{Contoh WebKit 8: Full Browser Application}

Program untuk mendemonstrasikan aplikasi browser lengkap.

\begin{lstlisting}[language=c++, caption=Full Browser Application]
\lstinputlisting[language=c++]{../code/13-qt-webkit-code-9.c++}
\end{lstlisting}

\section{Program GUI}

\subsection{Contoh GUI 1: Window Sederhana}

Program untuk mendemonstrasikan window sederhana.

\begin{lstlisting}[language=c++, caption=Window Sederhana]
\lstinputlisting[language=c++]{../code/14-gui-code-1.c++}
\end{lstlisting}

\subsection{Contoh GUI 2: Button dan Label}

Program untuk mendemonstrasikan button dan label.

\begin{lstlisting}[language=c++, caption=Button dan Label]
\lstinputlisting[language=c++]{../code/14-gui-code-2.c++}
\end{lstlisting}

\subsection{Contoh GUI 3: Layout Management}

Program untuk mendemonstrasikan layout management.

\begin{lstlisting}[language=c++, caption=Layout Management]
\lstinputlisting[language=c++]{../code/14-gui-code-3.c++}
\end{lstlisting}

\subsection{Contoh GUI 4: Signal dan Slot}

Program untuk mendemonstrasikan signal dan slot dalam GUI.

\begin{lstlisting}[language=c++, caption=Signal dan Slot dalam GUI]
\lstinputlisting[language=c++]{../code/14-gui-code-4.c++}
\end{lstlisting}

\section{Kesimpulan}

Bab ini berisi kumpulan lengkap contoh program C++ yang mencakup berbagai konsep pemrograman. Program-program ini dapat digunakan sebagai:

\begin{itemize}
\item \textbf{Referensi} - Untuk memahami implementasi konsep tertentu
\item \textbf{Latihan} - Untuk mempraktikkan konsep yang dipelajari
\item \textbf{Template} - Sebagai dasar untuk membuat program baru
\item \textbf{Belajar} - Untuk memahami best practices dalam pemrograman
\end{itemize}

Semua program dalam bab ini telah diuji dan dapat dijalankan dengan Qt Creator. Pembaca dianjurkan untuk:

\begin{enumerate}
\item \textbf{Mempelajari} setiap program secara teliti
\item \textbf{Memodifikasi} program sesuai kebutuhan
\item \textbf{Menggabungkan} konsep dari berbagai program
\item \textbf{Membuat} program baru berdasarkan contoh yang ada
\end{enumerate}

\begin{center}
\textbf{Selamat belajar dan happy coding!} 🚀
\end{center}